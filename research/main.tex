\documentclass{article}

% if you need to pass options to natbib, use, e.g.:
%     \PassOptionsToPackage{numbers, compress}{natbib}
% before loading neurips_2024


% ready for submission

% to compile a preprint version, e.g., for submission to arXiv, add add the
% [preprint] option:
%     \usepackage[preprint]{neurips_2024}


% to compile a camera-ready version, add the [final] option, e.g.:
%     \usepackage[final]{neurips_2024}
% to avoid loading the natbib package, add option nonatbib:
%    \usepackage[nonatbib]{neurips_2024}


\usepackage[utf8]{inputenc} % allow utf-8 input
\usepackage[T1]{fontenc}    % use 8-bit T1 fonts
\usepackage{hyperref}       % hyperlinks
\usepackage{url}            % simple URL typesetting
\usepackage{booktabs}       % professional-quality tables
\usepackage{amsfonts}       % blackboard math symbols
\usepackage{nicefrac}       % compact symbols for 1/2, etc.
\usepackage{microtype}      % microtypography
\usepackage{xcolor}         % colors


\title{Formatting Instructions For NeurIPS 2024}


% The \author macro works with any number of authors. There are two commands
% used to separate the names and addresses of multiple authors: \And and \AND.
%
% Using \And between authors leaves it to LaTeX to determine where to break the
% lines. Using \AND forces a line break at that point. So, if LaTeX puts 3 of 4
% authors names on the first line, and the last on the second line, try using
% \AND instead of \And before the third author name.


\author{%
  David S.~Hippocampus\thanks{Use footnote for providing further information
    about author (webpage, alternative address)---\emph{not} for acknowledging
    funding agencies.} \\
  Department of Computer Science\\
  Cranberry-Lemon University\\
  Pittsburgh, PA 15213 \\
  \texttt{hippo@cs.cranberry-lemon.edu} \\
  % examples of more authors
  % \And
  % Coauthor \\
  % Affiliation \\
  % Address \\
  % \texttt{email} \\
  % \AND
  % Coauthor \\
  % Affiliation \\
  % Address \\
  % \texttt{email} \\
  % \And
  % Coauthor \\
  % Affiliation \\
  % Address \\
  % \texttt{email} \\
  % \And
  % Coauthor \\
  % Affiliation \\
  % Address \\
  % \texttt{email} \\
}


\begin{document}


\maketitle


\begin{abstract}
  With accurate solutions to the many electron Schrodinger equation all the chemistry
  could derived from first principles. Try to find analytical is prohibitively hard due
  the intrinsic relations between each component on a molecule.
  In this work I develop the use of a architecture based on the Transformer 
  architecture to tackle this problem.
\end{abstract}


\section{Introduction}

The success of deep Learning across different fields like protein folding  @jumper2021highly, visual modeling @dosovitskiy2021imageworth16x16words, ODEs solvers @RAISSI2019686 has sparked great interest from the scientific community to apply DL methods to their fields. 

Quantum Chemistry, specifically finding a good aproximmation for the Quantum Many-Body wave eqaution  the is one of those places where have shown that deep learning could overpass traditional methods @Luo_2019 , @Qiao_2020, but there is still many challenges specifically, the computational power needed for large molecules becomes prohibitively expensive. 

Tackling that problem the Transformer architecture had demonstrate that scaling laws are not so much complicated for him. Cite

Motivated for that in this work I develop a transformer architecture called Psifomer. @vonglehn2023selfattentionansatzabinitioquantum  

\section{Objectives}

- Obtain a model which is able to replicate the energy ground states of certain atoms.
- Compare our model with another State of the art methods to solve the Many electrons Schrodinger equation


I provide an outline of the model architecture and procedure 
In theoretical frame work we  will and in methodology we will 


\section{Theoretical Framework}

In order to solve the problem we have to grasp the physics laws that our solution have to follow, 


We consider the follow:

\subsection{The Schrodinger Equation}

The schrodinger equation was presented in a series of publication made it by Schrodinger in the year 1916.

It was received pretty well by the scientific comunnity. 

And its relevance is high, in principle it is able to explain all the atomic phenomena and all the facts of chemical bindings.

\subsection{The many electron Schrodinger Equation}

In quantum chemistry is regular used atomic units, the unit of distance is the Bohr Radious and the unit of energy is Hartree (Ha).

In its time-independent form the Schrodinger equation can be written as a eigenfunction equation.


$$ \hat{H}\psi(\mathbf{x}_{0},\dots ,\mathbf{x}_{n})=E\psi(\mathbf{x}_{1},\dots ,\mathbf{x}_{n}) $$

Where $\hat{H}$ is a Hermitian linear operator called the Hamiltonian and the scalar eigenvalue $E$ corresponds to the energy of a particular solution.

$$
U=\frac{1}{4\pi\varepsilon_{0}}\frac{e^{2}}{\lvert r_{i}-r_{j} \rvert }
$$

Using atomic units we [[Quantum Chemistry units|Atomic Units]]:

The Hamiltonian using the [[Quantum Chemistry units]] becomes:

$$ \hat{H}=-\frac{1}{2}\sum \nabla^{2}+\sum \frac{1}{\lvert r_{i}-r_{j} \rvert }-\sum \frac{Z_{I}}{\lvert r_{i}-R_{I} \rvert }+\sum \frac{Z_{I}Z_{J}}{\lvert R_{i}-R_{j} \rvert } $$

Where $Z_{I}$ are the [[atomic number]] $r_{i}$ is the distance from a reference frame 

Now the [[Fermi Dirac Statistics]] tell us that this solution of this equation should be **anti symmetric** this is:

$$
\psi(\dots,\mathbf{x}_{i},\dots,\mathbf{x}_{j},\dots)=-\psi(\dots ,\mathbf{x}_{j},\dots ,\mathbf{x}_{i},\dots)
$$

The potential energy becomes infinite when two electrons overlap , this could be formalized via the [[Kato Cusp Conditions]], a Jastrow factor $\exp(\mathcal{J})$. The explicit form of $\mathcal{J}$ depends on the author.




\subsection{Approximating a solution}

Find possible solution in the traditional way is prohibitively hard. So what people have doing and it seem that it becomes a success is guess that solution and using another techniques to improve the solution, to this guess solution we called **Ansatz**.

Once that you have your Ansatz, which normally depends on depends on certain parameters.



\subVariational{Monte Carlo}

Once that you guess an **Ansatz** you optimize using the **rayleight quotient**.

$$ \mathcal{L}=\frac{\bra{\psi} \hat{H}\ket{\psi} }{\braket{ \psi | \psi } }=\frac{\int d\mathbf{r}\psi ^{*}(\mathbf{r})\hat{H}\psi(\mathbf{r})}{\int d\mathbf{r}\psi ^{*}(\mathbf{r})\psi(\mathbf{r})} $$


So how we optimized this. Here appears [[Variational Quantum Monte Carlo]].

Which can be re-written as:
$$ E_{L}(x)=\Psi ^{-1}_{\theta}(x)\hat{H}\Psi_{\theta}(x) $$

$$ \mathcal{L}_{\theta}=\mathbb{E}_{x\sim \Psi^{2}_{\theta}}[E_{L}(x)] $$

And here we use [[Metropolis algorithm]] to work in real life.

## Using Deep Learning

They are a quite example of it.

examples @shangSolvingManyelectronSchrodinger2025 Related work

### Neural Networksee

### RNN


Fermi Net

A very important work for us is: Fermi Net @Pfau_2020  it uses different MLP to learn the forms of the orbitals. Their ansatz is: [[Fermi Net]]

$$ \psi(\mathbf{x}_{i},\dots,\mathbf{x}_{n})=\sum_{k}\omega_{k}\det[\Phi ^{k}] $$

With:

$$
\begin{vmatrix}
\phi_{1}^{k}(\mathbf{x}_{1})  & \dots  &  \phi_{1}^{k}(\mathbf{x}_{n}) \\
\vdots   &  & \vdots  \\
\phi_{n}^{k}(\mathbf{x}_{1}) & \dots & \phi_{n}^{k}(\mathbf{x}_{n})

\end{vmatrix}=\det[\phi_{i}^{k}(\mathbf{x}_{j})]=\det[\Phi ^{k}]
$$

The elements of the determinant are obtained via

$\alpha \in \{ \uparrow,\downarrow \}$

$$
\mathbf{h}_{i}^{\ell \alpha} \gets \text{concatenate}(\mathbf{r}^\alpha_i - \mathbf{R}_I, |\mathbf{r}^\alpha_i - \mathbf{R}_I|\ \forall\ I)
$$
$$
\mathbf{h}_{ij}^{\ell \alpha\beta} \gets \text{concatenate}(\mathbf{r}^\alpha_i - \mathbf{r}^\beta_j, |\mathbf{r}^\alpha_i - \mathbf{r}^\beta_j|\ \forall\ j,\beta)
$$

$$
 \begin{align}
    &\left(
    \mathbf{h}^{\ell\alpha}_i,
    \frac{1}{n^\uparrow}\sum_{j=1}^{n^\uparrow} \mathbf{h}^{\ell\uparrow}_j, \frac{1}{n^\downarrow} \sum_{j=1}^{n^\downarrow} \mathbf{h}^{\ell\downarrow}_j,
    \frac{1}{n^\uparrow} \sum_{j=1}^{n^\uparrow} \mathbf{h}^{\ell\alpha\uparrow}_{ij},
    \frac{1}{n^\downarrow} \sum_{j=1}^{n^\downarrow} \mathbf{h}^{\ell\alpha\downarrow}_{ij}\right) \nonumber \\
    &\qquad =
    \left(\mathbf{h}^{\ell\alpha}_i, \mathbf{g}^{\ell\uparrow}, \mathbf{g}^{\ell\downarrow}, \mathbf{g}^{\ell\alpha\uparrow}_i, \mathbf{g}^{\ell\alpha\downarrow}_i\right) = \mathbf{f}^{\ell \alpha}_i,
\end{align}
$$


$$
\begin{align}
    \mathbf{h}^{\ell+1 \alpha}_i &= \mathrm{tanh}\left(\mathbf{V}^\ell \mathbf{f}^{\ell \alpha}_i + \mathbf{b}^\ell\right) + \mathbf{h}^{\ell\alpha}_i \nonumber \\
    \mathbf{h}^{\ell+1 \alpha\beta}_{ij} &= \mathrm{tanh}\left(\mathbf{W}^\ell\mathbf{h}^{\ell \alpha\beta}_{ij} + \mathbf{c}^\ell\right) + \mathbf{h}^{\ell \alpha\beta}_{ij}
\end{align}
$$

$$
\begin{multline}
    \phi^{k\alpha}_i(\mathbf{r}^\alpha_j; \{\mathbf{r}^\alpha_{/j}\}; \{\mathbf{r}^{\bar{\alpha}}\}) =
    \left(\mathbf{w}^{k\alpha}_i \cdot \mathbf{h}^{L\alpha}_j + g^{k\alpha}_i\right)\\
	\sum_{m} \pi^{k\alpha}_{im}\mathrm{exp}\left(-|\mathbf{\Sigma}_{im}^{k \alpha}(\mathbf{r}^{\alpha}_j-\mathbf{R}_m)|\right),
\end{multline}
$$

$$ \phi ^{k\alpha}_{i}(\mathbf{r}^{\alpha}_{j};\{ \mathbf{r}^{\alpha}_{/j} \};\{ \mathbf{r}^{\bar{\alpha}} \})=(\mathbf{w}^{k\alpha}_{i}\cdot \mathbf{h}^{L\alpha}_{j}+g^{k\alpha}_{i})\sum_{m}\pi_{im}^{k\alpha}\exp\left( -\left\lvert \Sigma _{im}^{k\alpha}(\mathbf{r}^{\alpha}_{j}-\mathbf{R}_{m})\right\rvert  \right)$$.

$$
​\begin{align}
	\psi(\mathbf{r}^\uparrow_1,\ldots,\mathbf{r}^\downarrow_{n^\downarrow}) = \sum_{k}\omega_k &\left(\det\left[\phi^{k \uparrow}_i(\mathbf{r}^\uparrow_j; \{\mathbf{r}^\uparrow_{/j}\}; \{\mathbf{r}^\downarrow\})\right]\right.\\&\left.\hphantom{\left(\right.}\det\left[\phi^{k\downarrow}_i(\mathbf{r}^\downarrow_j; \{\mathbf{r}^\downarrow_{/j}\});
	\{\mathbf{r}^\uparrow\};\right]\right).
\end{align}
$$

You com

![[ferminet.png|280x315]]

Motivated for the antisymmetry and the Kato cusp conditions our **Ansatz** take the form of: [
### Transformers

There exist several architectures that I can use Recurrent Neural Network, Long Short Term Memory. 

@Vaswani2017 

Recurrent Neural Network are: [[Recurrent Neural Network]]
And long short term memory are: [[Long Short Memory]]

Why on earth I would use [[Transformer]]? They are extremely good finding relations between its elements. And the best is that scale well due its [[Transform Architecture]]

Attention mechanism appear with @bahdanau2014neural but it didn't work so:

- [[Attention mechanism]]
- [[Self attention mechanism on one head]]
- [[Multi-head attention]]
$$
\mathbf{o}_{t,i}=\sum_{j=1}^{t}\text{Softmax}\left( \frac{\mathbf{q}^{T}_{t,i}\mathbf{k}_{j,i}}{\sqrt{ d_{h} }} \right) \mathbf{v}_{j,i}
$$
$$
\mathbf{u}_{t}=W^{O}[\mathbf{o}_{t,1};\mathbf{o}_{t,2};\dots ;\mathbf{o}_{t,n_{h}}]
$$

# Psi Former

[[Psi Former Ansatz]]. @vonglehn2023selfattentionansatzabinitioquantum
$$ \Psi_{\theta}(\mathbf{x})=\exp(\mathcal{J}_{\theta}(\mathbf{x}))\sum_{k=1}^{N_{\det}}\det[\boldsymbol{\Phi}_{\theta}^{k}(x)] $$

Where $\mathcal{J}_{\theta}$ is the [[Jastrow Factor for si Former]] and $\Phi$ are [[orbital for neural network fermi net|orbitals]]. 


Where $\mathcal{J}_{\theta}:(\mathbb{R}^{3}\times \{ \uparrow,\downarrow \})^{n}\to \mathbb{R}$

- So the question is how you define the outputs of that functions:
- [[Jastrow Factor]]
$$
\mathcal{J}_{\theta}(x)=\sum_{i<j;\sigma_{i}=\sigma_{j}}-\frac{1}{4}\frac{\alpha^{2}_{par}}{\alpha_{par}+\lvert \mathbf{r}_{i}-\mathbf{r}_{j} \rvert }+\sum_{i,j;\sigma_{i}\neq \sigma_{j}}-\frac{1}{2}\frac{\alpha^{2}_{anti}}{\alpha_{anti}+\lvert \mathbf{r}_{i}-\mathbf{r}_{j} \rvert }
$$

Architecture

![[psiformer.png|271x339]]

### Loss function

We are going to take the [[Rayleigh Quotient like Expectation Value]] like loss function.

### Optimizer 

[[Kroenecker factored Approximate Curvature]]

### Flow of the architecture

First compute:
$$ v_{h}=[\text{SelfAttn}(\mathbf{h}^{l}_{1},\dots,\mathbf{h}^{\ell}_{N};\mathbf{W}^{\ell h}_{q},\mathbf{W}^{\ell h}_{k},\mathbf{W}^{\ell h}_{v})] $$

Start with:

$$\mathbf{W}_{o}^{\ell}\text{concat}_{h}[\text{SelfAttn}(\mathbf{h}^{l}_{1},\dots,\mathbf{h}^{\ell}_{N};\mathbf{W}^{\ell h}_{q},\mathbf{W}^{\ell h}_{k},\mathbf{W}^{\ell h}_{v})]$$

With it you can obtain you hidden states, and then how you use it



With them you create the [[orbital for neural network fermi net]]

And you have it.

# Methodology

To implement the code, the choose of the library is important.

The three options to implement this kind of matter are JAX, Tensor Flow and pytorch, each one with his advantages and disadvantages.

## Environment

For this project we are going to be using Pytorch due his user-friendly and support. Python. with UV

## Training

Due the high computational power needed we are going to using GPUS and of course CUDA.

Is clear that we are going to use virtual GPUS, for that matter we have two option or well use a GPU via SSH or directly using services like Azure , Colab, or anothers matters.

The election of the GPU is not trivial. use TPUS are not a bad idea.



\clearpage
\clearpage
\clearpage



begin{abstract}
  The abstract paragraph should be indented \nicefrac{1}{2}~inch (3~picas) on
  both the left- and right-hand margins. Use 10~point type, with a vertical
  spacing (leading) of 11~points.  The word \textbf{Abstract} must be centered,
  bold, and in point size 12. Two line spaces precede the abstract. The abstract
  must be limited to one paragraph.
\end{abstract}



\section{Submission of papers to NeurIPS 2024}


Please read the instructions below carefully and follow them faithfully.


\subsection{Style}


Papers to be submitted to NeurIPS 2024 must be prepared according to the
instructions presented here. Papers may only be up to {\bf nine} pages long,
including figures. Additional pages \emph{containing only acknowledgments and
references} are allowed. Papers that exceed the page limit will not be
reviewed, or in any other way considered for presentation at the conference.


The margins in 2024 are the same as those in previous years.


Authors are required to use the NeurIPS \LaTeX{} style files obtainable at the
NeurIPS website as indicated below. Please make sure you use the current files
and not previous versions. Tweaking the style files may be grounds for
rejection.









\subsection{Retrieval of style files}


The style files for NeurIPS and other conference information are available on
the website at
\begin{center}
  \url{http://www.neurips.cc/}
\end{center}
The file \verb+neurips_2024.pdf+ contains these instructions and illustrates the
various formatting requirements your NeurIPS paper must satisfy.


The only supported style file for NeurIPS 2024 is \verb+neurips_2024.sty+,
rewritten for \LaTeXe{}.  \textbf{Previous style files for \LaTeX{} 2.09,
  Microsoft Word, and RTF are no longer supported!}


The \LaTeX{} style file contains three optional arguments: \verb+final+, which
creates a camera-ready copy, \verb+preprint+, which creates a preprint for
submission to, e.g., arXiv, and \verb+nonatbib+, which will not load the
\verb+natbib+ package for you in case of package clash.


\paragraph{Preprint option}
If you wish to post a preprint of your work online, e.g., on arXiv, using the
NeurIPS style, please use the \verb+preprint+ option. This will create a
nonanonymized version of your work with the text ``Preprint. Work in progress.''
in the footer. This version may be distributed as you see fit, as long as you do not say which conference it was submitted to. Please \textbf{do
  not} use the \verb+final+ option, which should \textbf{only} be used for
papers accepted to NeurIPS.


At submission time, please omit the \verb+final+ and \verb+preprint+
options. This will anonymize your submission and add line numbers to aid
review. Please do \emph{not} refer to these line numbers in your paper as they
will be removed during generation of camera-ready copies.


The file \verb+neurips_2024.tex+ may be used as a ``shell'' for writing your
paper. All you have to do is replace the author, title, abstract, and text of
the paper with your own.


The formatting instructions contained in these style files are summarized in
Sections \ref{gen_inst}, \ref{headings}, and \ref{others} below.


\section{General formatting instructions}
\label{gen_inst}


The text must be confined within a rectangle 5.5~inches (33~picas) wide and
9~inches (54~picas) long. The left margin is 1.5~inch (9~picas).  Use 10~point
type with a vertical spacing (leading) of 11~points.  Times New Roman is the
preferred typeface throughout, and will be selected for you by default.
Paragraphs are separated by \nicefrac{1}{2}~line space (5.5 points), with no
indentation.


The paper title should be 17~point, initial caps/lower case, bold, centered
between two horizontal rules. The top rule should be 4~points thick and the
bottom rule should be 1~point thick. Allow \nicefrac{1}{4}~inch space above and
below the title to rules. All pages should start at 1~inch (6~picas) from the
top of the page.


For the final version, authors' names are set in boldface, and each name is
centered above the corresponding address. The lead author's name is to be listed
first (left-most), and the co-authors' names (if different address) are set to
follow. If there is only one co-author, list both author and co-author side by
side.


Please pay special attention to the instructions in Section \ref{others}
regarding figures, tables, acknowledgments, and references.


\section{Headings: first level}
\label{headings}


All headings should be lower case (except for first word and proper nouns),
flush left, and bold.


First-level headings should be in 12-point type.


\subsection{Headings: second level}


Second-level headings should be in 10-point type.


\subsubsection{Headings: third level}


Third-level headings should be in 10-point type.


\paragraph{Paragraphs}


There is also a \verb+\paragraph+ command available, which sets the heading in
bold, flush left, and inline with the text, with the heading followed by 1\,em
of space.


\section{Citations, figures, tables, references}
\label{others}


These instructions apply to everyone.


\subsection{Citations within the text}


The \verb+natbib+ package will be loaded for you by default.  Citations may be
author/year or numeric, as long as you maintain internal consistency.  As to the
format of the references themselves, any style is acceptable as long as it is
used consistently.


The documentation for \verb+natbib+ may be found at
\begin{center}
  \url{http://mirrors.ctan.org/macros/latex/contrib/natbib/natnotes.pdf}
\end{center}
Of note is the command \verb+\citet+, which produces citations appropriate for
use in inline text.  For example,
\begin{verbatim}
   \citet{hasselmo} investigated\dots
\end{verbatim}
produces
\begin{quote}
  Hasselmo, et al.\ (1995) investigated\dots
\end{quote}


If you wish to load the \verb+natbib+ package with options, you may add the
following before loading the \verb+neurips_2024+ package:
\begin{verbatim}
   \PassOptionsToPackage{options}{natbib}
\end{verbatim}


If \verb+natbib+ clashes with another package you load, you can add the optional
argument \verb+nonatbib+ when loading the style file:
\begin{verbatim}
   \usepackage[nonatbib]{neurips_2024}
\end{verbatim}


As submission is double blind, refer to your own published work in the third
person. That is, use ``In the previous work of Jones et al.\ [4],'' not ``In our
previous work [4].'' If you cite your other papers that are not widely available
(e.g., a journal paper under review), use anonymous author names in the
citation, e.g., an author of the form ``A.\ Anonymous'' and include a copy of the anonymized paper in the supplementary material.


\subsection{Footnotes}


Footnotes should be used sparingly.  If you do require a footnote, indicate
footnotes with a number\footnote{Sample of the first footnote.} in the
text. Place the footnotes at the bottom of the page on which they appear.
Precede the footnote with a horizontal rule of 2~inches (12~picas).


Note that footnotes are properly typeset \emph{after} punctuation
marks.\footnote{As in this example.}


\subsection{Figures}


\begin{figure}
  \centering
  \fbox{\rule[-.5cm]{0cm}{4cm} \rule[-.5cm]{4cm}{0cm}}
  \caption{Sample figure caption.}
\end{figure}


All artwork must be neat, clean, and legible. Lines should be dark enough for
purposes of reproduction. The figure number and caption always appear after the
figure. Place one line space before the figure caption and one line space after
the figure. The figure caption should be lower case (except for first word and
proper nouns); figures are numbered consecutively.


You may use color figures.  However, it is best for the figure captions and the
paper body to be legible if the paper is printed in either black/white or in
color.


\subsection{Tables}


All tables must be centered, neat, clean and legible.  The table number and
title always appear before the table.  See Table~\ref{sample-table}.


Place one line space before the table title, one line space after the
table title, and one line space after the table. The table title must
be lower case (except for first word and proper nouns); tables are
numbered consecutively.


Note that publication-quality tables \emph{do not contain vertical rules.} We
strongly suggest the use of the \verb+booktabs+ package, which allows for
typesetting high-quality, professional tables:
\begin{center}
  \url{https://www.ctan.org/pkg/booktabs}
\end{center}
This package was used to typeset Table~\ref{sample-table}.


\begin{table}
  \caption{Sample table title}
  \label{sample-table}
  \centering
  \begin{tabular}{lll}
    \toprule
    \multicolumn{2}{c}{Part}                   \\
    \cmidrule(r){1-2}
    Name     & Description     & Size ($\mu$m) \\
    \midrule
    Dendrite & Input terminal  & $\sim$100     \\
    Axon     & Output terminal & $\sim$10      \\
    Soma     & Cell body       & up to $10^6$  \\
    \bottomrule
  \end{tabular}
\end{table}

\subsection{Math}
Note that display math in bare TeX commands will not create correct line numbers for submission. Please use LaTeX (or AMSTeX) commands for unnumbered display math. (You really shouldn't be using \$\$ anyway; see \url{https://tex.stackexchange.com/questions/503/why-is-preferable-to} and \url{https://tex.stackexchange.com/questions/40492/what-are-the-differences-between-align-equation-and-displaymath} for more information.)

\subsection{Final instructions}

Do not change any aspects of the formatting parameters in the style files.  In
particular, do not modify the width or length of the rectangle the text should
fit into, and do not change font sizes (except perhaps in the
\textbf{References} section; see below). Please note that pages should be
numbered.


\section{Preparing PDF files}


Please prepare submission files with paper size ``US Letter,'' and not, for
example, ``A4.''


Fonts were the main cause of problems in the past years. Your PDF file must only
contain Type 1 or Embedded TrueType fonts. Here are a few instructions to
achieve this.


\begin{itemize}


\item You should directly generate PDF files using \verb+pdflatex+.


\item You can check which fonts a PDF files uses.  In Acrobat Reader, select the
  menu Files$>$Document Properties$>$Fonts and select Show All Fonts. You can
  also use the program \verb+pdffonts+ which comes with \verb+xpdf+ and is
  available out-of-the-box on most Linux machines.


\item \verb+xfig+ "patterned" shapes are implemented with bitmap fonts.  Use
  "solid" shapes instead.


\item The \verb+\bbold+ package almost always uses bitmap fonts.  You should use
  the equivalent AMS Fonts:
\begin{verbatim}
   \usepackage{amsfonts}
\end{verbatim}
followed by, e.g., \verb+\mathbb{R}+, \verb+\mathbb{N}+, or \verb+\mathbb{C}+
for $\mathbb{R}$, $\mathbb{N}$ or $\mathbb{C}$.  You can also use the following
workaround for reals, natural and complex:
\begin{verbatim}
   \newcommand{\RR}{I\!\!R} %real numbers
   \newcommand{\Nat}{I\!\!N} %natural numbers
   \newcommand{\CC}{I\!\!\!\!C} %complex numbers
\end{verbatim}
Note that \verb+amsfonts+ is automatically loaded by the \verb+amssymb+ package.


\end{itemize}


If your file contains type 3 fonts or non embedded TrueType fonts, we will ask
you to fix it.


\subsection{Margins in \LaTeX{}}


Most of the margin problems come from figures positioned by hand using
\verb+\special+ or other commands. We suggest using the command
\verb+\includegraphics+ from the \verb+graphicx+ package. Always specify the
figure width as a multiple of the line width as in the example below:
\begin{verbatim}
   \usepackage[pdftex]{graphicx} ...
   \includegraphics[width=0.8\linewidth]{myfile.pdf}
\end{verbatim}
See Section 4.4 in the graphics bundle documentation
(\url{http://mirrors.ctan.org/macros/latex/required/graphics/grfguide.pdf})


A number of width problems arise when \LaTeX{} cannot properly hyphenate a
line. Please give LaTeX hyphenation hints using the \verb+\-+ command when
necessary.

\begin{ack}
Use unnumbered first level headings for the acknowledgments. All acknowledgments
go at the end of the paper before the list of references. Moreover, you are required to declare
funding (financial activities supporting the submitted work) and competing interests (related financial activities outside the submitted work).
More information about this disclosure can be found at: \url{https://neurips.cc/Conferences/2024/PaperInformation/FundingDisclosure}.


Do {\bf not} include this section in the anonymized submission, only in the final paper. You can use the \texttt{ack} environment provided in the style file to automatically hide this section in the anonymized submission.
\end{ack}

\section*{References}


References follow the acknowledgments in the camera-ready paper. Use unnumbered first-level heading for
the references. Any choice of citation style is acceptable as long as you are
consistent. It is permissible to reduce the font size to \verb+small+ (9 point)
when listing the references.
Note that the Reference section does not count towards the page limit.
\medskip


{
\small


[1] Alexander, J.A.\ \& Mozer, M.C.\ (1995) Template-based algorithms for
connectionist rule extraction. In G.\ Tesauro, D.S.\ Touretzky and T.K.\ Leen
(eds.), {\it Advances in Neural Information Processing Systems 7},
pp.\ 609--616. Cambridge, MA: MIT Press.


[2] Bower, J.M.\ \& Beeman, D.\ (1995) {\it The Book of GENESIS: Exploring
  Realistic Neural Models with the GEneral NEural SImulation System.}  New York:
TELOS/Springer--Verlag.


[3] Hasselmo, M.E., Schnell, E.\ \& Barkai, E.\ (1995) Dynamics of learning and
recall at excitatory recurrent synapses and cholinergic modulation in rat
hippocampal region CA3. {\it Journal of Neuroscience} {\bf 15}(7):5249-5262.
}


%%%%%%%%%%%%%%%%%%%%%%%%%%%%%%%%%%%%%%%%%%%%%%%%%%%%%%%%%%%%

\appendix

\section{Appendix / supplemental material}


Optionally include supplemental material (complete proofs, additional experiments and plots) in appendix.
All such materials \textbf{SHOULD be included in the main submission.}

%%%%%%%%%%%%%%%%%%%%%%%%%%%%%%%%%%%%%%%%%%%%%%%%%%%%%%%%%%%%

\newpage
\section*{NeurIPS Paper Checklist}

%%% BEGIN INSTRUCTIONS %%%
The checklist is designed to encourage best practices for responsible machine learning research, addressing issues of reproducibility, transparency, research ethics, and societal impact. Do not remove the checklist: {\bf The papers not including the checklist will be desk rejected.} The checklist should follow the references and follow the (optional) supplemental material.  The checklist does NOT count towards the page
limit.

Please read the checklist guidelines carefully for information on how to answer these questions. For each question in the checklist:
\begin{itemize}
    \item You should answer \answerYes{}, \answerNo{}, or \answerNA{}.
    \item \answerNA{} means either that the question is Not Applicable for that particular paper or the relevant information is Not Available.
    \item Please provide a short (1–2 sentence) justification right after your answer (even for NA).
   % \item {\bf The papers not including the checklist will be desk rejected.}
\end{itemize}

{\bf The checklist answers are an integral part of your paper submission.} They are visible to the reviewers, area chairs, senior area chairs, and ethics reviewers. You will be asked to also include it (after eventual revisions) with the final version of your paper, and its final version will be published with the paper.

The reviewers of your paper will be asked to use the checklist as one of the factors in their evaluation. While "\answerYes{}" is generally preferable to "\answerNo{}", it is perfectly acceptable to answer "\answerNo{}" provided a proper justification is given (e.g., "error bars are not reported because it would be too computationally expensive" or "we were unable to find the license for the dataset we used"). In general, answering "\answerNo{}" or "\answerNA{}" is not grounds for rejection. While the questions are phrased in a binary way, we acknowledge that the true answer is often more nuanced, so please just use your best judgment and write a justification to elaborate. All supporting evidence can appear either in the main paper or the supplemental material, provided in appendix. If you answer \answerYes{} to a question, in the justification please point to the section(s) where related material for the question can be found.

IMPORTANT, please:
\begin{itemize}
    \item {\bf Delete this instruction block, but keep the section heading ``NeurIPS paper checklist"},
    \item  {\bf Keep the checklist subsection headings, questions/answers and guidelines below.}
    \item {\bf Do not modify the questions and only use the provided macros for your answers}.
\end{itemize}


%%% END INSTRUCTIONS %%%


\begin{enumerate}

\item {\bf Claims}
    \item[] Question: Do the main claims made in the abstract and introduction accurately reflect the paper's contributions and scope?
    \item[] Answer: \answerTODO{} % Replace by \answerYes{}, \answerNo{}, or \answerNA{}.
    \item[] Justification: \justificationTODO{}
    \item[] Guidelines:
    \begin{itemize}
        \item The answer NA means that the abstract and introduction do not include the claims made in the paper.
        \item The abstract and/or introduction should clearly state the claims made, including the contributions made in the paper and important assumptions and limitations. A No or NA answer to this question will not be perceived well by the reviewers.
        \item The claims made should match theoretical and experimental results, and reflect how much the results can be expected to generalize to other settings.
        \item It is fine to include aspirational goals as motivation as long as it is clear that these goals are not attained by the paper.
    \end{itemize}

\item {\bf Limitations}
    \item[] Question: Does the paper discuss the limitations of the work performed by the authors?
    \item[] Answer: \answerTODO{} % Replace by \answerYes{}, \answerNo{}, or \answerNA{}.
    \item[] Justification: \justificationTODO{}
    \item[] Guidelines:
    \begin{itemize}
        \item The answer NA means that the paper has no limitation while the answer No means that the paper has limitations, but those are not discussed in the paper.
        \item The authors are encouraged to create a separate "Limitations" section in their paper.
        \item The paper should point out any strong assumptions and how robust the results are to violations of these assumptions (e.g., independence assumptions, noiseless settings, model well-specification, asymptotic approximations only holding locally). The authors should reflect on how these assumptions might be violated in practice and what the implications would be.
        \item The authors should reflect on the scope of the claims made, e.g., if the approach was only tested on a few datasets or with a few runs. In general, empirical results often depend on implicit assumptions, which should be articulated.
        \item The authors should reflect on the factors that influence the performance of the approach. For example, a facial recognition algorithm may perform poorly when image resolution is low or images are taken in low lighting. Or a speech-to-text system might not be used reliably to provide closed captions for online lectures because it fails to handle technical jargon.
        \item The authors should discuss the computational efficiency of the proposed algorithms and how they scale with dataset size.
        \item If applicable, the authors should discuss possible limitations of their approach to address problems of privacy and fairness.
        \item While the authors might fear that complete honesty about limitations might be used by reviewers as grounds for rejection, a worse outcome might be that reviewers discover limitations that aren't acknowledged in the paper. The authors should use their best judgment and recognize that individual actions in favor of transparency play an important role in developing norms that preserve the integrity of the community. Reviewers will be specifically instructed to not penalize honesty concerning limitations.
    \end{itemize}

\item {\bf Theory Assumptions and Proofs}
    \item[] Question: For each theoretical result, does the paper provide the full set of assumptions and a complete (and correct) proof?
    \item[] Answer: \answerTODO{} % Replace by \answerYes{}, \answerNo{}, or \answerNA{}.
    \item[] Justification: \justificationTODO{}
    \item[] Guidelines:
    \begin{itemize}
        \item The answer NA means that the paper does not include theoretical results.
        \item All the theorems, formulas, and proofs in the paper should be numbered and cross-referenced.
        \item All assumptions should be clearly stated or referenced in the statement of any theorems.
        \item The proofs can either appear in the main paper or the supplemental material, but if they appear in the supplemental material, the authors are encouraged to provide a short proof sketch to provide intuition.
        \item Inversely, any informal proof provided in the core of the paper should be complemented by formal proofs provided in appendix or supplemental material.
        \item Theorems and Lemmas that the proof relies upon should be properly referenced.
    \end{itemize}

    \item {\bf Experimental Result Reproducibility}
    \item[] Question: Does the paper fully disclose all the information needed to reproduce the main experimental results of the paper to the extent that it affects the main claims and/or conclusions of the paper (regardless of whether the code and data are provided or not)?
    \item[] Answer: \answerTODO{} % Replace by \answerYes{}, \answerNo{}, or \answerNA{}.
    \item[] Justification: \justificationTODO{}
    \item[] Guidelines:
    \begin{itemize}
        \item The answer NA means that the paper does not include experiments.
        \item If the paper includes experiments, a No answer to this question will not be perceived well by the reviewers: Making the paper reproducible is important, regardless of whether the code and data are provided or not.
        \item If the contribution is a dataset and/or model, the authors should describe the steps taken to make their results reproducible or verifiable.
        \item Depending on the contribution, reproducibility can be accomplished in various ways. For example, if the contribution is a novel architecture, describing the architecture fully might suffice, or if the contribution is a specific model and empirical evaluation, it may be necessary to either make it possible for others to replicate the model with the same dataset, or provide access to the model. In general. releasing code and data is often one good way to accomplish this, but reproducibility can also be provided via detailed instructions for how to replicate the results, access to a hosted model (e.g., in the case of a large language model), releasing of a model checkpoint, or other means that are appropriate to the research performed.
        \item While NeurIPS does not require releasing code, the conference does require all submissions to provide some reasonable avenue for reproducibility, which may depend on the nature of the contribution. For example
        \begin{enumerate}
            \item If the contribution is primarily a new algorithm, the paper should make it clear how to reproduce that algorithm.
            \item If the contribution is primarily a new model architecture, the paper should describe the architecture clearly and fully.
            \item If the contribution is a new model (e.g., a large language model), then there should either be a way to access this model for reproducing the results or a way to reproduce the model (e.g., with an open-source dataset or instructions for how to construct the dataset).
            \item We recognize that reproducibility may be tricky in some cases, in which case authors are welcome to describe the particular way they provide for reproducibility. In the case of closed-source models, it may be that access to the model is limited in some way (e.g., to registered users), but it should be possible for other researchers to have some path to reproducing or verifying the results.
        \end{enumerate}
    \end{itemize}


\item {\bf Open access to data and code}
    \item[] Question: Does the paper provide open access to the data and code, with sufficient instructions to faithfully reproduce the main experimental results, as described in supplemental material?
    \item[] Answer: \answerTODO{} % Replace by \answerYes{}, \answerNo{}, or \answerNA{}.
    \item[] Justification: \justificationTODO{}
    \item[] Guidelines:
    \begin{itemize}
        \item The answer NA means that paper does not include experiments requiring code.
        \item Please see the NeurIPS code and data submission guidelines (\url{https://nips.cc/public/guides/CodeSubmissionPolicy}) for more details.
        \item While we encourage the release of code and data, we understand that this might not be possible, so “No” is an acceptable answer. Papers cannot be rejected simply for not including code, unless this is central to the contribution (e.g., for a new open-source benchmark).
        \item The instructions should contain the exact command and environment needed to run to reproduce the results. See the NeurIPS code and data submission guidelines (\url{https://nips.cc/public/guides/CodeSubmissionPolicy}) for more details.
        \item The authors should provide instructions on data access and preparation, including how to access the raw data, preprocessed data, intermediate data, and generated data, etc.
        \item The authors should provide scripts to reproduce all experimental results for the new proposed method and baselines. If only a subset of experiments are reproducible, they should state which ones are omitted from the script and why.
        \item At submission time, to preserve anonymity, the authors should release anonymized versions (if applicable).
        \item Providing as much information as possible in supplemental material (appended to the paper) is recommended, but including URLs to data and code is permitted.
    \end{itemize}


\item {\bf Experimental Setting/Details}
    \item[] Question: Does the paper specify all the training and test details (e.g., data splits, hyperparameters, how they were chosen, type of optimizer, etc.) necessary to understand the results?
    \item[] Answer: \answerTODO{} % Replace by \answerYes{}, \answerNo{}, or \answerNA{}.
    \item[] Justification: \justificationTODO{}
    \item[] Guidelines:
    \begin{itemize}
        \item The answer NA means that the paper does not include experiments.
        \item The experimental setting should be presented in the core of the paper to a level of detail that is necessary to appreciate the results and make sense of them.
        \item The full details can be provided either with the code, in appendix, or as supplemental material.
    \end{itemize}

\item {\bf Experiment Statistical Significance}
    \item[] Question: Does the paper report error bars suitably and correctly defined or other appropriate information about the statistical significance of the experiments?
    \item[] Answer: \answerTODO{} % Replace by \answerYes{}, \answerNo{}, or \answerNA{}.
    \item[] Justification: \justificationTODO{}
    \item[] Guidelines:
    \begin{itemize}
        \item The answer NA means that the paper does not include experiments.
        \item The authors should answer "Yes" if the results are accompanied by error bars, confidence intervals, or statistical significance tests, at least for the experiments that support the main claims of the paper.
        \item The factors of variability that the error bars are capturing should be clearly stated (for example, train/test split, initialization, random drawing of some parameter, or overall run with given experimental conditions).
        \item The method for calculating the error bars should be explained (closed form formula, call to a library function, bootstrap, etc.)
        \item The assumptions made should be given (e.g., Normally distributed errors).
        \item It should be clear whether the error bar is the standard deviation or the standard error of the mean.
        \item It is OK to report 1-sigma error bars, but one should state it. The authors should preferably report a 2-sigma error bar than state that they have a 96\% CI, if the hypothesis of Normality of errors is not verified.
        \item For asymmetric distributions, the authors should be careful not to show in tables or figures symmetric error bars that would yield results that are out of range (e.g. negative error rates).
        \item If error bars are reported in tables or plots, The authors should explain in the text how they were calculated and reference the corresponding figures or tables in the text.
    \end{itemize}

\item {\bf Experiments Compute Resources}
    \item[] Question: For each experiment, does the paper provide sufficient information on the computer resources (type of compute workers, memory, time of execution) needed to reproduce the experiments?
    \item[] Answer: \answerTODO{} % Replace by \answerYes{}, \answerNo{}, or \answerNA{}.
    \item[] Justification: \justificationTODO{}
    \item[] Guidelines:
    \begin{itemize}
        \item The answer NA means that the paper does not include experiments.
        \item The paper should indicate the type of compute workers CPU or GPU, internal cluster, or cloud provider, including relevant memory and storage.
        \item The paper should provide the amount of compute required for each of the individual experimental runs as well as estimate the total compute.
        \item The paper should disclose whether the full research project required more compute than the experiments reported in the paper (e.g., preliminary or failed experiments that didn't make it into the paper).
    \end{itemize}

\item {\bf Code Of Ethics}
    \item[] Question: Does the research conducted in the paper conform, in every respect, with the NeurIPS Code of Ethics \url{https://neurips.cc/public/EthicsGuidelines}?
    \item[] Answer: \answerTODO{} % Replace by \answerYes{}, \answerNo{}, or \answerNA{}.
    \item[] Justification: \justificationTODO{}
    \item[] Guidelines:
    \begin{itemize}
        \item The answer NA means that the authors have not reviewed the NeurIPS Code of Ethics.
        \item If the authors answer No, they should explain the special circumstances that require a deviation from the Code of Ethics.
        \item The authors should make sure to preserve anonymity (e.g., if there is a special consideration due to laws or regulations in their jurisdiction).
    \end{itemize}


\item {\bf Broader Impacts}
    \item[] Question: Does the paper discuss both potential positive societal impacts and negative societal impacts of the work performed?
    \item[] Answer: \answerTODO{} % Replace by \answerYes{}, \answerNo{}, or \answerNA{}.
    \item[] Justification: \justificationTODO{}
    \item[] Guidelines:
    \begin{itemize}
        \item The answer NA means that there is no societal impact of the work performed.
        \item If the authors answer NA or No, they should explain why their work has no societal impact or why the paper does not address societal impact.
        \item Examples of negative societal impacts include potential malicious or unintended uses (e.g., disinformation, generating fake profiles, surveillance), fairness considerations (e.g., deployment of technologies that could make decisions that unfairly impact specific groups), privacy considerations, and security considerations.
        \item The conference expects that many papers will be foundational research and not tied to particular applications, let alone deployments. However, if there is a direct path to any negative applications, the authors should point it out. For example, it is legitimate to point out that an improvement in the quality of generative models could be used to generate deepfakes for disinformation. On the other hand, it is not needed to point out that a generic algorithm for optimizing neural networks could enable people to train models that generate Deepfakes faster.
        \item The authors should consider possible harms that could arise when the technology is being used as intended and functioning correctly, harms that could arise when the technology is being used as intended but gives incorrect results, and harms following from (intentional or unintentional) misuse of the technology.
        \item If there are negative societal impacts, the authors could also discuss possible mitigation strategies (e.g., gated release of models, providing defenses in addition to attacks, mechanisms for monitoring misuse, mechanisms to monitor how a system learns from feedback over time, improving the efficiency and accessibility of ML).
    \end{itemize}

\item {\bf Safeguards}
    \item[] Question: Does the paper describe safeguards that have been put in place for responsible release of data or models that have a high risk for misuse (e.g., pretrained language models, image generators, or scraped datasets)?
    \item[] Answer: \answerTODO{} % Replace by \answerYes{}, \answerNo{}, or \answerNA{}.
    \item[] Justification: \justificationTODO{}
    \item[] Guidelines:
    \begin{itemize}
        \item The answer NA means that the paper poses no such risks.
        \item Released models that have a high risk for misuse or dual-use should be released with necessary safeguards to allow for controlled use of the model, for example by requiring that users adhere to usage guidelines or restrictions to access the model or implementing safety filters.
        \item Datasets that have been scraped from the Internet could pose safety risks. The authors should describe how they avoided releasing unsafe images.
        \item We recognize that providing effective safeguards is challenging, and many papers do not require this, but we encourage authors to take this into account and make a best faith effort.
    \end{itemize}

\item {\bf Licenses for existing assets}
    \item[] Question: Are the creators or original owners of assets (e.g., code, data, models), used in the paper, properly credited and are the license and terms of use explicitly mentioned and properly respected?
    \item[] Answer: \answerTODO{} % Replace by \answerYes{}, \answerNo{}, or \answerNA{}.
    \item[] Justification: \justificationTODO{}
    \item[] Guidelines:
    \begin{itemize}
        \item The answer NA means that the paper does not use existing assets.
        \item The authors should cite the original paper that produced the code package or dataset.
        \item The authors should state which version of the asset is used and, if possible, include a URL.
        \item The name of the license (e.g., CC-BY 4.0) should be included for each asset.
        \item For scraped data from a particular source (e.g., website), the copyright and terms of service of that source should be provided.
        \item If assets are released, the license, copyright information, and terms of use in the package should be provided. For popular datasets, \url{paperswithcode.com/datasets} has curated licenses for some datasets. Their licensing guide can help determine the license of a dataset.
        \item For existing datasets that are re-packaged, both the original license and the license of the derived asset (if it has changed) should be provided.
        \item If this information is not available online, the authors are encouraged to reach out to the asset's creators.
    \end{itemize}

\item {\bf New Assets}
    \item[] Question: Are new assets introduced in the paper well documented and is the documentation provided alongside the assets?
    \item[] Answer: \answerTODO{} % Replace by \answerYes{}, \answerNo{}, or \answerNA{}.
    \item[] Justification: \justificationTODO{}
    \item[] Guidelines:
    \begin{itemize}
        \item The answer NA means that the paper does not release new assets.
        \item Researchers should communicate the details of the dataset/code/model as part of their submissions via structured templates. This includes details about training, license, limitations, etc.
        \item The paper should discuss whether and how consent was obtained from people whose asset is used.
        \item At submission time, remember to anonymize your assets (if applicable). You can either create an anonymized URL or include an anonymized zip file.
    \end{itemize}

\item {\bf Crowdsourcing and Research with Human Subjects}
    \item[] Question: For crowdsourcing experiments and research with human subjects, does the paper include the full text of instructions given to participants and screenshots, if applicable, as well as details about compensation (if any)?
    \item[] Answer: \answerTODO{} % Replace by \answerYes{}, \answerNo{}, or \answerNA{}.
    \item[] Justification: \justificationTODO{}
    \item[] Guidelines:
    \begin{itemize}
        \item The answer NA means that the paper does not involve crowdsourcing nor research with human subjects.
        \item Including this information in the supplemental material is fine, but if the main contribution of the paper involves human subjects, then as much detail as possible should be included in the main paper.
        \item According to the NeurIPS Code of Ethics, workers involved in data collection, curation, or other labor should be paid at least the minimum wage in the country of the data collector.
    \end{itemize}

\item {\bf Institutional Review Board (IRB) Approvals or Equivalent for Research with Human Subjects}
    \item[] Question: Does the paper describe potential risks incurred by study participants, whether such risks were disclosed to the subjects, and whether Institutional Review Board (IRB) approvals (or an equivalent approval/review based on the requirements of your country or institution) were obtained?
    \item[] Answer: \answerTODO{} % Replace by \answerYes{}, \answerNo{}, or \answerNA{}.
    \item[] Justification: \justificationTODO{}
    \item[] Guidelines:
    \begin{itemize}
        \item The answer NA means that the paper does not involve crowdsourcing nor research with human subjects.
        \item Depending on the country in which research is conducted, IRB approval (or equivalent) may be required for any human subjects research. If you obtained IRB approval, you should clearly state this in the paper.
        \item We recognize that the procedures for this may vary significantly between institutions and locations, and we expect authors to adhere to the NeurIPS Code of Ethics and the guidelines for their institution.
        \item For initial submissions, do not include any information that would break anonymity (if applicable), such as the institution conducting the review.
    \end{itemize}

\end{enumerate}



\section{Development of a Transformed based architecture to solve the
Time Independent Many Electron Schrodinger
Equation}\label{development-of-a-transformed-based-architecture-to-solve-the-time-independent-many-electron-schrodinger-equation}

\section{Abstract}\label{abstract}

With accurate solutions to the many-electron Schrodinger equation all
the chemistry could be derived from first principles, but analytical
treatment is intractable due the intrinsic strong electron-electron
correlations, anti symmetry and cusp behavior. Recently, due to its high
flexibility deep learning approaches had been applied for this problem,
neural wave function models such as FermiNet and PauliNet are two good
examples, these have advanced accuracy, yet computational cost and error
typically grows steeply with system size, limiting applicability to
larger molecules. They also lack of strong architectures designed to
capture long-range electronic correlations with scalable attention. In
this work I develop the Psiformer a transformer-based ansatz that
couples scalable attention with physics-aware structure. Training is
formulated within Variational Monte Carlo (VMC), evaluation will be do
it by comparing against another traditional methods, I also outline
design questions for further improvement, including sparsified/global
attention and optimizer choices inspired by recent transformer advances.
\# Introduction

The electronic structure problem remains challenging: the wave functions
which describes fully the systems lives in a \(3N\)-dimensional space
(\(N\) number of electrons, each one lives on the 3 dimensional space,
interesting molecules have around 10 electrons), additionally it must
satisfy certain properties due to physical laws plus that the solutions
lives on the complex space.

Although the governing laws have been known for a century---see
Schrödinger's formulation of wave mechanics {[}Schrödinger,
1926{]}---obtaining practical approximations to the quantum many-body
wavefunction remains difficult. Established approaches such as
density-functional theory, the Born--Oppenheimer separation, and
structured variational ansätze trade generality for tractability by
imposing specific functional forms or approximations to correlation.
These choices are effective within their regimes but can struggle for
strongly correlated systems or scale poorly with electron count. This is
where modern learning-based methods enter: instead of fixing the
functional form, we learn it, while enforcing essential physics
(antisymmetry, cusp behavior, permutation symmetry).

Deep learning has reshaped several scientific domains, from protein
structure prediction @jumper2021highly to vision
@dosovitskiy2021imageworth16x16words and PDE surrogates @RAISSI2019686.
Motivated by these successes, the community has explored neural
approaches for quantum many-body problems, seeking accurate and scalable
approximations to the many-electron wavefunction.

Thus neural wavefunction models have emerged as a promising alternative.
Architectures such as \textbf{FermiNet} and \textbf{PauliNet} combine
flexible function approximators with determinant structures to respect
antisymmetry, improving variational expressivity. However, two practical
limitations persist. First, error or compute cost often scales
unfavorably with the number of electrons, restricting applicability to
larger molecules. Second, mechanisms for \textbf{long-range electronic
correlation}---central to Coulomb and exchange effects---are typically
implicit or expensive to capture, leading to optimization difficulty and
brittle generalization.

Transformers offer an appealing direction. Self-attention provides
direct, many-to-many interactions among tokens in a single layer, is
highly parallelizable, and has empirically favorable scaling behavior in
other domains (Natural Language Processing) . For electronic structure,
where any electron can interact with any other and electron indices are
exchangeable, attention aligns naturally with the physics: it enables
global coupling without imposing an arbitrary ordering. The challenge is
to embed the right \textbf{inductive biases} (distance awareness, spin
structure, cusp handling) and to maintain \textbf{fermionic
antisymmetry} while keeping computational cost under control.

I develop \textbf{Psiformer}, a transformer-based variational ansatz for
many-electron systems. Psiformer uses self-attention to construct rich
per-electron features informed by electron--electron and
electron--nucleus descriptors, then imposes antisymmetry explicitly via
determinant-based heads. Physics-aware priors (e.g., distance/radial
encodings and cusp-motivated embeddings) are incorporated to reduce
sample complexity and stabilize training. The optimization is formulated
within \textbf{Variational Monte Carlo (VMC)} by minimizing the
variational energy.

\textbf{Contributions.} 1. A transformer-based neural wavefunction
(\textbf{Psiformer}) that separates correlation modeling (attention)
from fermionic symmetry (determinant heads) while embedding
Coulomb-aware priors. 2. A VMC training recipe with practical choices
for stability (feature design, damping, and optional natural-gradient
preconditioning).

@vonglehn2023selfattentionansatzabinitioquantum\\
@Luo\_2019 @Qiao\_2020.

The objectives are the follow: \# Objectives

\begin{itemize}
\tightlist
\item
  Obtain a model which is able to approximate the ground state state
  energy of the carbon atom.
\item
  Compare our model with another state of the art methods to solve the
  many electrons Schrodinger equation respect the ground state energy.
\item
  Look for future improvements when try to tackle larger molecules. \#
  Overview
\end{itemize}

This work is structured as follow: The theoretical framework introduces
the foundations of quantum many-body theory, the structure of the
Schrodinger equation for many-bodies like also foundational concepts of
Deep Learning that are going to be used in the development of this
specific problem.

Section {[}Number{]} introduce \textbf{Psi Former} a transformer based
architecture built upon \textbf{Fermi Net}. Section {[}Number{]} talk
about the Methodology which is going to be used to implement this work.
\# Theoretical Framework

The concepts presented in this section provideS the physical and
mathematical background for our specific problem, introduces the physics
of the wave function like also the necessary deep learning concepts.
\#\# The physics law behind the solution

All the physics and methods related necessary. \#\#\# The Schrodinger
Equation

The Schrodinger equation was presented in a series of publications made
it by Erwin Schrodinger in the year 1926. There we search the complex
function \(\psi\) which lives on a Hilbert space \(\mathcal{H}\) called
\textbf{wave function}, for a non relativistic spinless single particle
this function depends on the position of the particle
\(\mathbf{\vec{r}}\) and time \(t\) \((\psi(\mathbf{\vec{r}},t))\), the
quantity \(\lvert \psi (\mathbf{r},t)\rvert^{2}\) is the
\textbf{probability density} to find the particle near \(\mathbf{r}\) at
time \(t\).

Guided by de Broglie's discover of the wave particle duality and a very
smart intuition Schrodinger proposed the time dependent equation (TDSE):
\[ i\hbar \frac{\partial \psi}{\partial t}=\hat{H}\psi \] Where \(i\) is
the complex unit, \(\hbar\) is the {[}{[}Reduced Planck Constant{]}{]}
approximate to \(1.054571817\dots \times 10^{-34} J\cdot s\) and
\(\hat{H}\) is a Hermitian linear operator called the
\textbf{Hamiltonian} which represents the total energy of the system,
for a single non relativistic (low energy) particle of mass \(m\) in a
scalar potential \(V(\mathbf{r},t)\). \[
\hat{H}=\frac{\hat{\mathbf{p}}^{2}}{2m}+V(\mathbf{r},t)
\] Where \(\mathbf{\hat{p}}\) is the Momentum Operator and in the
\textbf{Position representation} it takes the form of: \[
\mathbf{\hat{p}}=-i\hbar \nabla
\] Where \(\nabla\) is the Laplacian operator, thus the TDSE is
explicitly: \[
i\hbar\,\frac{\partial \psi}{\partial t}
=
\left[-\frac{\hbar^{2}}{2m}\nabla^{2}+V(\mathbf r,t)\right]\psi
\] The \textbf{time independent form} (TISE) could be derived from
equation ,when the wave function \(\psi\) could be written like the
product of two functions \(R\) and \(T\), where \(R\) depends uniquely
on the spatial term \((\mathbf{\vec{r}})\) and \(T\) uniquely on the
temporal \((t)\), this is: \[
\psi(\mathbf{\vec{r}},t)=R(\mathbf{\vec{r}})T(t)
\] Plugin this form on equation \textbf{number}, you can derive that: \[
T(t)=e^{ -iEt/\hbar }
\] Where \(E\),the energy of the system, is constant. You also obtain
that the spatial part is conditioned by: \[
\hat{H}R(\mathbf{\vec{r}})=ER(\mathbf{\vec{r}})
\] We are going to represent the spatial function \(R\) as \(\psi\). And
in this work we are going to only focus in the TDSE, when we treat with
constant energy, the electron are almost always found near the lowest
energy state, known as the ground state. Solutions with higher energy
known as excited states are relevant to photochemistry , but in this
work we will restrict our attention to ground states. \#\#\# The many
electron Schrodinger Equation When we are considering more than one
single particle we consider the spin \((\sigma)\) and the interaction
between particles. Thus in its time-independent form the Schrodinger
equation can be written as an eigen value problem:
\[ \hat{H}\psi(\mathbf{x}_{0},\dots ,\mathbf{x}_{n})=E\psi(\mathbf{x}_{1},\dots ,\mathbf{x}_{n}) \]
Where \(\mathbf{x}_{i}=\{ \mathbf{r}_{i},\sigma \}\),
\(\mathbf{r}_{i}\in \mathbb{R}^{3}\) is the position of each particle
and \(\sigma \in \{ \uparrow.\downarrow \}\) is the so called spin. It's
possible model the potential energy of a many body system (e.g atoms,
molecules), we first have to consider the repulsion between each
electrons: \[
V_{ij}
= \frac{e^{2}}{4\pi\varepsilon_{0}}
  \frac{1}{\lvert \mathbf{r}_{i}-\mathbf{r}_{j} \rvert}
\] Here: - \(e = 1.602\,176\,634\times 10^{-19}\ \mathrm{C}\) is the
elementary charge, -
\(\varepsilon_{0} = 8.854\,187\,8128\times 10^{-12}\ \mathrm{F\,m^{-1}}\)
is the \textbf{electrical permittivity of vacuum}, - \(\mathbf{r}_i\) is
the position vector of electron \(i\) in the chosen reference frame. The
attraction between the proton \(I\) and the electron \(i\) is given by:
\[
V_{iI} = -\frac{1}{4\pi\varepsilon_{0}} \frac{eZ_{I}}{\lvert \mathbf{r}_{i} - \mathbf{R}_{I} \rvert}
\] Where \(Z_I\) is the atomic number of nucleus \(I\) (for instance, in
a Helium atom \(Z = 2\)) and \(\mathbf{R}_I\) is the position of that
nucleus from a chosen reference frame.\\
The reference frame is usually taken at the \textbf{center of mass} or
at the \textbf{center of the molecule}. The repulsion between the nuclei
\(I\) and \(J\) (protons) is: \[
V_{IJ} = \frac{1}{4\pi\varepsilon_{0}} \frac{Z_{I}Z_{J}}{\lvert \mathbf{R}_{I} - \mathbf{R}_{J} \rvert}
\] To avoid writing these constants every time, quantum chemistry
commonly uses \textbf{atomic units (a.u.)}. In this system, the unit of
length is the \textbf{Bohr radius} \(a_{0}\),\\
and the unit of energy is the \textbf{Hartree} \(E_{h}\): \[
E_{h} = \frac{e^{2}}{4\pi\varepsilon_{0} a_{0}}
\] Under atomic units, \(e = 1\),
\(4\pi\varepsilon_{0} = 1\),\(\hbar = 1\), and \(m_{e} = 1\).\\
Thus, the potential energy of a system with \(N_{e}\) electron and
\(N_{n}\) nucleus can be written compactly as: \[
V = -\sum_{i=1}^{N_{e}}\sum_{I=1}^{N_{n}}  \frac{Z_I}{\lvert \mathbf{r}_i - \mathbf{R}_I \rvert}
+ \sum_{i>j} \frac{1}{\lvert \mathbf{r}_i - \mathbf{r}_j \rvert}
+ \sum_{I>J} \frac{Z_I Z_J}{\lvert \mathbf{R}_I - \mathbf{R}_J \rvert}
\]

where: - \(i, j\) index electrons. - \(I, J\) index nuclei. - The first
term represents \textbf{electron--nucleus attraction}. - The second term
is \textbf{electron--electron repulsion}. - The third term is
\textbf{nucleus--nucleus repulsion}. For the kinetic term we need to
consider two expressions: \(\nabla_i^2\) acts on the \textbf{electron}
coordinates \(\mathbf r_i\) (fast, light particles) and \(\nabla_I^2\)
acts on the \textbf{nuclear} coordinates \(\mathbf R_I\) (slow, heavy
particles). Let \(N\) electrons at \(\mathbf r_i\) and \(M\) nuclei at
\(\mathbf R_I\) with charges \(Z_I\) and masses \(M_I\) (in units of
\(m_e\)) then the \textbf{Hamiltonian} can be written as: \[
\boxed{
\hat H =
-\sum_{i=1}^{N}\frac{1}{2}\nabla_i^{2}
-\sum_{I=1}^{M}\frac{1}{2M_I}\nabla_{I}^{2}
-\sum_{i=1}^{N}\sum_{I=1}^{M}\frac{Z_I}{|\mathbf r_i-\mathbf R_I|}
+\sum_{1\le i<j\le N}\frac{1}{|\mathbf r_i-\mathbf r_j|}
+\sum_{1\le I<J\le M}\frac{Z_I Z_J}{|\mathbf R_I-\mathbf R_J|}
}
\] \#\#\# Conditions of the solution

As with any differential equation, where one is searching one solution
is important to consider initial conditions (IC) and boundary conditions
(BC), here we have to fulfill certain conditions that comes from
physical laws. \#\#\#\# Fermi--Dirac statistics and Pauli Exclusion
Bosons (e.g., photons) follow \textbf{Bose--Einstein} statistics,
whereas fermions (e.g., electrons, protons) follow \textbf{Fermi--Dirac}
statistics, because at the microscopic level identical particles are
indistinguishable then the wave function should be equal but Pauli
Exclusion don't like that , the many-particle wave function must be
\textbf{anti symmetric} under the exchange of any two fermions. This
implies: \[
\psi(\dots,\mathbf{x}_{i},\dots,\mathbf{x}_{j},\dots)=-\psi(\dots ,\mathbf{x}_{j},\dots ,\mathbf{x}_{i},\dots)
\] We can enforce \textbf{antisymmetry} using a \(N\times N\)
determinant, which involves one-particle states only (a wave function
with a single input) . An interchange of any pair of particles
corresponds to an interchange of two columns of the determinant; this
interchange introduces a change in the sign of the determinant. For even
permutations we have \((-1)^{P}=1\), and for odd permutations we have
\((-1)^{P}=-1\). {[}{[}Fermi Dirac Statistics{]}{]}

\[
\Psi(\mathbf x_1,\ldots,\mathbf x_N)
\propto
\begin{vmatrix}
\phi_1(\mathbf x_1) & \phi_2(\mathbf x_1) & \cdots & \phi_N(\mathbf x_1)\\
\phi_1(\mathbf x_2) & \phi_2(\mathbf x_2) & \cdots & \phi_N(\mathbf x_2)\\
\vdots & \vdots & \ddots & \vdots\\
\phi_1(\mathbf x_N) & \phi_2(\mathbf x_N) & \cdots & \phi_N(\mathbf x_N)
\end{vmatrix},
\] Where \(\phi_k\) are orthonormal (by construction)
{[}{[}spin-orbital\textbar spin orbitals{]}{]}. This is, for instance
\((N=2)\): \[
\Psi(\mathbf x_1,\mathbf x_2)
\propto\Big[\phi_a(\mathbf x_1)\phi_b(\mathbf x_2)-\phi_a(\mathbf x_2)\phi_b(\mathbf x_1)\Big].
\] \#\#\#\# Kato Cusp Conditions

The potential energy becomes infinite, when particles overlap, which
places strict constraints on the form of the wave function at these
points, knows as the \textbf{Kato Cusp Conditions} {[}cite{]}. The cusp
conditions states that the wave function must be non-differentiable at
these points, and give exact values for the average derivatives at the
cusps. This can be obtained if we multiply to the ansatz by multiplying
by a Jastrow factor \(\mathcal{J}\)which satisfies these conditions
analytically. {[}cite{]}. - Electron-nucleus cusp (electron with charge
\(-1\), nucleus charge \(+Z\), reduced mass \(\mu \approx 1\))
\[\lim_{ riI \to 0 } \left( \frac{\partial \psi}{\partial r_{iI}} \right)=-Z\psi(r_{iI}=0)
\] - Electron-electron cusp, opposite spins (charges \(-1\), reduced
mass \(\mu=\frac{1}{2}\)) \[
\lim_{ r_{ij}\to 0 } \left( \frac{\partial \psi}{\partial r_{ij}} \right)=\frac{1}{2}\psi(r_{ij}=0)
\] Where \(r_{iI}(r_{ij})\) is an electron-nuclear (electron-electron)
distance, \(Z_{I}\) is the nuclear charge of the \(I\text{-th}\)
nucleous and ave implies a spherical averaging over all directions.
{[}{[}Kato Cusp Conditions{]}{]}.

So that is the problem we need to find a \(\psi\) such that it satisfies
all those \#\#\# Approximations to the problem

Is clear that find analytical solutions is practically impossible, so
what people have been doing is first apply good approximations.

\textbf{Born Oppenheimer} approximation makes it possible to separate
the motion of the nuclei and the motion of the electrons, neglects the
motion of the atomic nuclei when describing the electrons in a molecule.
The physical basis for the Born-Oppenheimer approximation is the fact
that the mass of an atomic nucleus in a molecule is much larger than the
mass of an electron (more than 1000 times). Because of this difference,
the nuclei move much more slowly than the electrons. In addition, due to
their opposite charges, there is a mutual attractive force of: acting on
an atomic nucleus and an electron. This force causes both particles to
be accelerated. Since the magnitude of the acceleration is inversely
proportional to the mass, \(a = f/m\), the acceleration of the electrons
is large and the acceleration of the atomic nuclei is small; the
difference is a factor of more than 1000. Consequently, the electrons
are moving and responding to forces very quickly, and the nuclei are
not, then we fix the nucleus, \(\mathbf{R}_{I}\) becomes a constant,
thus the kinetic term for nucleus becomes zero. And the potential energy
for the repulsion between nucleus becomes a constant. \[
\boxed{
\hat H_{\mathrm{el}} =
-\sum_{i=1}^{N}\frac{1}{2}\nabla_i^{2}
-\sum_{i=1}^{N}\sum_{I=1}^{M}\frac{Z_I}{|\mathbf r_i-\mathbf R_I|}
+\sum_{1\le i<j\le N}\frac{1}{|\mathbf r_i-\mathbf r_j|}
+ \sum_{I<J}\frac{Z_I Z_J}{|\mathbf R_I-\mathbf R_J|}
}
\]

This is the form that we are going to work with; a second useful
approach to the problem is use an \textbf{Ansatz}, which is a guess
solution guided by intuition, this normally depend on certain number of
parameters, then the problem becomes on optimize this \textbf{Ansatz}.

Another important matter are the complex numbers, we should to work with
them, no! Why is that? \#\#\# Rayleigh Quotient We need a way to measure
how well our Ansatz \(\psi_{\theta}\) is doing, if our ansatz is bad
then don't reflect the truly form of the system, Deep learning approachs
use data sets to train their ansatz (model initialized on randoms
parameters) in this case we don't need any external but physical
principles. Lets first introduce the \textbf{Rayleigh quotient}. If
\(A\) is an operator and \(x\) is a state, the number: \[
R_{A}(x)=\frac{\braket{ x |A|x  }}{\braket{ x | x } }
\] is the \textbf{expectation value} of that operator in that state. The
important for us is that if \(\psi\) is a wave function and \(\hat{H}\)
the \textbf{Hamiltonian}, then the Rayleigh quotient: \[
R_{\hat{H}}(\psi)=\frac{\braket{ \psi | \hat{H}| \psi}}{\braket{ \psi | \psi } }
\] is the average (expected) energy of the system when it is in the
state \(\psi\). \#\#\# Variational Principle

The {[}{[}Variational Principle{]}{]} states that the expectation value
for the binding energy obtained using an approximate wave function and
the exact Hamiltonian operator will be higher than or equal to the true
energy for the system. This idea is really powerful. When implemented it
permits us to find the best approximate wavefunction from a given
wavefunction that contains one or more adjustable parameters, called a
trial wavefunction. A mathematical statement of the variational
principle is. \[
R_{\hat{H}}(\psi_{\text{ansatz}})\geq R_{\hat{H}}(\psi_{\text{true}})
\] The true ground-state wave function \(\psi_{\text{true}}\) is the one
that minimizes the rayleigh quotient. \[
\psi_{0}=\underset{\psi}{\text{argmin}}( R_{\hat{H}}(\psi))
\] So if we minimize the rayleigh quotient of our ansatz we are going to
be more near of the true wave function. \#\#\# Variational Monte Carlo

To the process that we are going to use for optimize our ansatz is
called Variational Monte Carlo (VMC). Now we can see to the rayleigh
quotient as a loss function \(\mathcal{L}\) with the form of. \[
\mathcal{L}(\Psi_{\theta})=\frac{\bra{\Psi_{\theta}} \hat{H}\ket{\Psi_{\theta}} }{\braket{ \Psi_{\theta} | \Psi _{\theta}} }=\frac{\int d\mathbf{r}\Psi ^{*}(\mathbf{r})\hat{H}\Psi(\mathbf{r})}{\int d\mathbf{r}\Psi ^{*}(\mathbf{r})\Psi(\mathbf{r})}
\] Evaluate that integral is hard, another smart approach is the follow,
define a probability distribution \(p_{\theta}\) with the follow form:
\[
p_{\theta}(x)=\frac{\lvert \Psi_{\theta} (x)\rvert ^{2}}{\int dx'\Psi^{2}_{\theta}(x')}
\] Realize that for compute \(p_{\theta}(x)\) for a specific \(x\) is
complicated due the integral that appears, this is going to be important
later. Defining the local energy \(E_{L}\) with: \[
E_{L}(x)=\Psi ^{-1}_{\theta}(x)\hat{H}\Psi_{\theta}(x)
\] Then the loss becomes: \[
\mathcal{L}(\Psi_{\theta})=\int \frac{ \hat{H}\Psi_{\theta}(x)}{\Psi_{\theta}(x)}p_{\theta}(x)dx
\] Which is the expected value of the local energy: \[
\mathcal{L}(\Psi_{\theta})=\mathbb{E}_{x\sim p_{\theta}}[E_{L}(x)]
\] To optimize our wave function we know by deep learning that we need
evaluate that expectation and obtain the derivative for back
propagation, how we make that? We are going to use the the Monte Carlo
estimatior: \[
\mathcal{L}_{\theta}=\mathbb{E}_{x\sim p_{\theta}}[E_{L}(x)]\approx \frac{1}{M}\sum_{i=1}^{M} E_{L}(\mathbf{R}_{k})
\] Whit \(\mathbf{R}_{k}\) are samples from the \(p_{\theta}\)
probability distribution. This is \[
\mathbf{R}_{1},\dots,\mathbf{R}_{M}\sim p_{\theta}(\mathbf{R})
\] We obtain \(E_{L}\) using: \[
E_{L}(\mathbf{R}_{k})=\frac{\hat{H}\psi(\mathbf{R}_{k})}{\psi(\mathbf{R}_{k})}=-\frac{1}{2}\frac{\nabla^{2}\psi(\mathbf{R_{k}})}{\psi(\mathbf{R}_{k})}+V(\mathbf{R}_{k})
\] Calculus tell us that for a any derivable function \(f\). \[
\frac{\nabla^{2}f}{f}=[\nabla^{2}\log f+(\nabla f)^{2}]
\] In practice is more numerically stable work using that form, thus: \[
E_{L}(\mathbf{R}_{k})=-\frac{1}{2}\sum_{i=1}^{N} \sum_{j=1}^{3} \left[ \frac{\partial ^{2} \log \lvert \Psi(x) \rvert }{\partial r_{ij}^{2}}+\left( \frac{\partial \log \lvert \Psi(x) \rvert }{\partial r_{ij}} \right)^{2} \right]+V(\mathbf{R_{k}})
\] The gradient of the energy respect to the parameters by a
parameterized wave functions is: \[
\nabla _{\theta}\mathcal{L}=2\mathbb{E}_{x\sim \Psi^{2}}[(E_{L}(x)-\mathbb{E}_{x'\sim\Psi^{2}}[E_{L}(x')])\nabla \log \lvert \Psi(x) \rvert ]
\] \#\#\# Metropolis Hastings (MH) Algorithm To obtain samples from the
probability distribution \(p_{\theta}\) we are going to use the
{[}{[}Metropolis Hasting algorithm{]}{]}. (MH)

MH is a {[}{[}Markov Chain Monte Carlo{]}{]} (MCMC) method used to
obtain a sequence of random samples from a probability distribution. The
reason to use this method over another well knows methods (e.g.~example)
is that MH don't suffer of the {[}{[}Curse of Dimensionality{]}{]} this
is, it remains strong while increase the dimension of the problem, and
since we are going to be working on dimension around ten is efficient
use this method. The algorithm works like follow:

\begin{enumerate}
\def\labelenumi{\arabic{enumi}.}
\tightlist
\item
  Take a initial configuration \(\mathbf{X}_{0}\in E\) arbitrary:
\item
  Propose \(\mathbf{X}'=\mathbf{X}_{0}+\eta\) ,where
  \(\eta \sim q(\eta)\), \(q\) is a probability density on \(E\) called
  \textbf{proposal kernel}. In our case we are going to a
  {[}{[}symmetric Gaussian{]}{]}.
\item
  Compute the quantity: \[
  A(\mathbf{X_{0}}, \mathbf{X}')=\text{min}\left( 1,\frac{\rho(\mathbf{X}')}{\rho(\mathbf{X}_{0})} \frac{q(\mathbf{X}'-\mathbf{X}_{0})}{q(\mathbf{X}_{0}-\mathbf{X}')}\right)
  \] Where \(\rho\) is the target distribution where we want sample, In
  the case where \(q\) is symmetric, this simplifies to: \[
  A(\mathbf{X}_{0},\mathbf{X}')=\text{min}\left( 1,\frac{\rho(\mathbf{X})}{\rho(\mathbf{X}_{0})} \right)
  \] Note that in our case \(\rho\) is equal to \(p_{\theta}\), we said
  that compute the integral factor remains a challenge, but in this case
  we have it. \[
  \frac{p_{\theta}(\mathbf{X}')}{p_{\theta}(\mathbf{X}_{0})}=\frac{\lvert \psi_{\theta}(\mathbf{X}') \rvert ^{2}/\int \lvert \psi_{\theta} \rvert ^{2}dx}{\lvert \psi_{\theta}(\mathbf{X}_{0}) \rvert ^{2}/\int \lvert \psi_{\theta} \rvert ^{2}dx}=\frac{\lvert \psi_{\theta}(\mathbf{X'}) \rvert ^{2}}{\lvert \psi_{\theta}(\mathbf{X}_{0}) \rvert ^{2}}
  \]
\item
  Generate a uniform number \(U\in[0,1]\)
\item
  If: \(U<A(\mathbf{X}_{0}\to \mathbf{X'}_{l})\) then
  \(\mathbf{X_{1}}=\mathbf{X}'\), otherwise try another \(\mathbf{X}'\).
  Accept or decline.
\item
  Repeat until obtain \(N_{eq}\) accepted sample, the changes stabilizes
  (we reach a stationary distribution) this phase is called \textbf{burn
  in}.
\item
  From \(\mathbf{X}_{N_{\text{eq}}}\) generate \(M\) samples until reach
  the sample \(\mathbf{X}_{N_{\text{eq}}+M+1}\). In each sample
  generates \(E_{L}(\mathbf{R}_{k})\) then average to obtain
  \(\mathbb{E}(E_{L})\) and with it and the derivative from equation
  number {[}number{]} we are able to begin the back propagation step.
  \#\# Deep Learning Fundamentals This subsection introduces the core
  concepts of Deep Learning that are going to be applied in this work.
  \#\#\# Multi Layer Perceptron A multi layer perceptron (MLP) is a
  nonlinear function
  \(\mathcal{F}:\mathbb{R}^{\text{in}}\to \mathbb{R}^{\text{out}}\).
  @nielsenNeuralNetworksDeep2015 , it actually is the composition of
  \(L\) layers, the first layer is called the input layer, the last
  output layer and the intermediates hidden layers. In each layer we
  find a arbitrary number of neurons although is a good practice always
  choose number which are powers of two and an affine map
  \(\mathbf{z}^{(l)},l\in \{ L,L-1,\dots,2 \}\) (\(l=1\) is the input
  layer) of the follow form. \[
  \mathbf{z}^{(l)}=\mathbf{W}^{(l)}\mathbf{a}^{(l-1)}+\mathbf{b}^{(l)}
  \] Where \(\mathbf{W}^{(l)}\) called weight matrix and
  \(\mathbf{b}^{(l)}\) the bias vector of the \(l\) layer. We use a
  non-linear function \(\sigma ^{(l)}\) in the \(l\) layer (typically
  Softmax, ReLu, Tanh), thus the output of each layer is:
  \[ f^{(l)}=\sigma ^{(l)}\circ \mathbf{z}^{(l)} \] Where \(\circ\)
  means composition. A MLP is the composition of all the layers. \[
  \mathcal{F}=f^{(L)}\circ f^{(L-1)}\circ\dots \circ f^{(1)}
  \] We call parameters to the set of all the weights and bias of each
  layer. And represented it with the symbol \(\theta\).
  \[\{ \mathbf{W}^{(l)},\mathbf{b}^{(l)}\}_{l=2}^{L}=\theta\] You
  typically train a MLP, using a training data set, a loss function (e.g
  Mean Square Error, Mean Absolute Error, Cross entropy) and an
  optimizer (e.g GD, SGD, ADAM). Additionally you can use regularization
  techniques such as dropout to improve the generalization of the Net.
\end{enumerate}

\subsubsection{Natural gradient Descent}\label{natural-gradient-descent}

As we mentioned, there are many ways to update the parameters of a
neural network: Gradient Descent, Stochastic Gradient Descent,
{[}{[}Adaptive Moment Estimation{]}{]} (ADAM), etc. All of them
implicitly assume that the parameter space
\(\Theta \subset \mathbb{R}^d\) is equipped with the standard Euclidean
metric, so that ``length'' and ``steepest descent'' are measured with
respect to \(\|\Delta\theta\|_2\).

In our case the loss \(\mathcal{L}(\theta)\) depends on a probability
distribution \(p_\theta\), not just on \(\theta\) directly. For example,
in variational Monte Carlo we take \[
p_\theta(x)
= \frac{|\psi_\theta(x)|^2}{\displaystyle \int |\psi_\theta(x')|^2\,dx'} ,
\] so \(\theta\) parametrizes an entire family of probability densities
over configurations \(x\). It is therefore more natural to measure
distances between \emph{distributions} \(p_\theta\) and
\(p_{\theta+\Delta\theta}\), rather than between the parameter vectors
themselves.

A canonical way to measure the distance between nearby probability
distributions is the Kullback--Leibler (KL) divergence \[
\mathrm{KL}\big(p_\theta \,\|\, p_{\theta+\Delta\theta}\big)
= \mathbb{E}_{x\sim p_\theta}\!\left[\log\frac{p_\theta(x)}{p_{\theta+\Delta\theta}(x)}\right].
\] For small steps \(\Delta\theta\) one can show that a second--order
Taylor expansion of the KL gives \[
\mathrm{KL}\big(p_\theta \,\|\, p_{\theta+\Delta\theta}\big)
= \tfrac12\,\Delta\theta^\top \mathcal{F}(\theta)\,\Delta\theta + \mathcal{O}(\|\Delta\theta\|^3),
\] where \(\mathcal{F}(\theta)\) is the Fisher Information Matrix (FIM).
To define it, introduce the \textbf{score function} \[
s_\theta(x) \in \mathbb{R}^d, \qquad
s_\theta(x) = \nabla_\theta \log p(x\mid \theta),
\] then the FIM is \[
\mathcal{F}(\theta)
= \mathbb{E}_{x\sim p(\cdot\mid\theta)}\!\big[\,s_\theta(x)\,s_\theta(x)^{\mathsf T}\big].
\]

The set of distributions \[
\mathcal{M} = \{\, p_\theta(z)\;|\; \theta \in \Theta \subset \mathbb{R}^d \,\}
\] can be viewed as a differentiable manifold, and
\(\mathcal{F}(\theta)\) defines a Riemannian metric on its tangent
space. Concretely, for tangent vectors \(u,v \in \mathbb{R}^d\) at
\(\theta\) we define the inner product \[
\langle u,v \rangle_\theta
= u^{\mathsf T}\,\mathcal{F}(\theta)\,v.
\] This metric says: two parameter directions are ``close'' if they
induce similar infinitesimal changes in the \emph{distribution}
\(p_\theta\).

Now ask the usual steepest--descent question, but with this
non-Euclidean metric:

Find the direction \(\Delta\theta\) that decreases
\(\mathcal{L}(\theta)\) the fastest, among all directions with fixed
``length''
\(\|\Delta\theta\|_\theta^2 = \Delta\theta^\top \mathcal{F}(\theta)\,\Delta\theta\).

Solving this constrained optimization problem (e.g.~with Lagrange
multipliers) yields the \textbf{natural gradient} direction \[
\Delta\theta_{\text{nat}} \;\propto\; -\,\mathcal{F}(\theta)^{-1}\,\nabla_\theta \mathcal{L}(\theta).
\] Thus the natural gradient descent update is \[
\Delta\theta_{\text{nat}}
= -\,\eta\,\mathcal{F}(\theta)^{-1}\,\nabla_\theta \mathcal{L}(\theta),
\] where \(\eta>0\) is a step size. Compared with the usual gradient
\(\nabla_\theta \mathcal{L}\), the factor \(\mathcal{F}^{-1}\)
``preconditions'' the gradient by the local geometry of the model's
probability distribution: directions that barely change \(p_\theta\) are
amplified, directions that change it a lot are damped.

Natural gradient descent is therefore meaningful exactly in the
situation we care about: when the loss depends on the parameters
\emph{through} a probability model \(p_\theta\) (e.g.~likelihood,
cross-entropy, KL, variational objectives, variational Monte Carlo
energy, etc.).

\subsubsection{Kronecker Factored Approximate
Curvature}\label{kronecker-factored-approximate-curvature}

Directly computing and inverting the full Fisher matrix
\(\mathcal{F}(\theta)\) is infeasible for modern neural networks, since
\(\theta\) can have millions of components. Kronecker Factored
Approximate Curvature (KFAC) is an efficient approximation that makes
natural gradient updates practical for layered networks.

We sketch the construction for a fully connected layer \(\ell\) with
weight matrix \(W_\ell\) and (for simplicity) no bias. Bias terms can be
included by augmenting the activations with a constant \(1\); we comment
on this below. \#\#\#\# Forward definition of \(\mathbf{a}_\ell\)

Consider a standard MLP. For a single input sample \(x\), the forward
pass at layer \(\ell\) is

\begin{itemize}
\item
  \textbf{Input (activation) to layer \(\ell\)}: \[
  \mathbf{a}_\ell \in \mathbb{R}^{n_\ell}
  \] This is the column vector of activations coming into layer
  \(\ell\). For the first hidden layer, \(\mathbf{a}_1\) is just the
  (possibly preprocessed) input. For deeper layers it is the
  nonlinearity output from the previous layer.
\item
  \textbf{Pre-activation at layer \(\ell\)}: \[
  \mathbf{h}_\ell = W_\ell \,\mathbf{a}_\ell,
  \] where \(W_\ell \in \mathbb{R}^{m_\ell \times n_\ell}\).
\item
  \textbf{Output activation of layer \(\ell\)}: \[
  \tilde{\mathbf{a}}_\ell = \phi(\mathbf{h}_\ell),
  \] where \(\phi\) is applied element-wise. In many notations
  \(\tilde{\mathbf{a}}_\ell\) would become the input to the next layer,
  but to keep notation consistent with the Fisher block for \(W_\ell\),
  we explicitly distinguish:

  \begin{itemize}
  \tightlist
  \item
    \(\mathbf{a}_\ell\): input to \(W_\ell\),
  \item
    \(\mathbf{h}_\ell\): pre-activation,
  \item
    \(\tilde{\mathbf{a}}_\ell\): output activation of layer \(\ell\).
  \end{itemize}
\end{itemize}

In KFAC, when we talk about \(\mathbf{a}_\ell\) for the Fisher block of
\(W_\ell\), we always mean ``the vector that \(W_\ell\) multiplies on
the right''. \#\#\#\# Backward definition of \(\mathbf{e}_\ell\) Let the
loss for a single sample be \(\mathcal{L}(\theta)\) (for example,
negative log-likelihood or negative log of the wave-function
probability). Define the \textbf{backward sensitivity} (or error signal)
at layer \(\ell\) as \[
\mathbf{e}_\ell
= \frac{\partial \mathcal{L}}{\partial \mathbf{h}_\ell} \in \mathbb{R}^{m_\ell}.
\] This is computed via backpropagation: - At the output layer \(L\): \[
  \mathbf{e}_L
  = \frac{\partial \mathcal{L}}{\partial \mathbf{h}_L}
  = \left(\frac{\partial \mathcal{L}}{\partial \tilde{\mathbf{a}}_L}\right) \odot \phi'(\mathbf{h}_L),
  \] where \(\odot\) is the element-wise product or also Hadamard
product. - For hidden layers \(\ell < L\): \[
  \mathbf{e}_\ell
  = \frac{\partial \mathcal{L}}{\partial \mathbf{h}_\ell}
  = \left(W_{\ell+1}^{\mathsf T} \mathbf{e}_{\ell+1}\right) \odot \phi'(\mathbf{h}_\ell).
\] In the context of natural gradient for probabilistic models,
\(\mathcal{L}\) is often chosen as \(-\log p(X\mid\theta)\), so up to a
sign we can also think of \(\mathbf{e}_\ell\) as \[
\mathbf{e}_\ell = \frac{\partial \log p(X\mid\theta)}{\partial \mathbf{h}_\ell}.
\] \#\#\#\# Gradient w.r.t. \(W_\ell\) and the form
\(\mathbf{a}_\ell \otimes \mathbf{e}_\ell\)

For a single sample, using the chain rule, \[
\frac{\partial \mathcal{L}}{\partial W_\ell}
= \frac{\partial \mathcal{L}}{\partial \mathbf{h}_\ell}
  \frac{\partial \mathbf{h}_\ell}{\partial W_\ell}
= \mathbf{e}_\ell\, \mathbf{a}_\ell^{\mathsf T}.
\]

If instead of \(\mathcal{L}\) we use \(\log p(X\mid\theta)\) (as in the
Fisher definition), we get \[
\frac{\partial \log p(X\mid\theta)}{\partial W_\ell}
= \mathbf{e}_\ell\, \mathbf{a}_\ell^{\mathsf T},
\] with
\(\mathbf{e}_\ell = \partial \log p / \partial \mathbf{h}_\ell\).

Now vectorize the gradient. Using the standard identity \[
\mathrm{vec}(uv^{\mathsf T}) = v \otimes u,
\] with \(u = \mathbf{e}_\ell\) and \(v = \mathbf{a}_\ell\), we obtain
\[
\frac{\partial \log p(X\mid\theta)}{\partial \mathrm{vec}(W_\ell)}
= \mathrm{vec}\!\left(\frac{\partial \log p}{\partial W_\ell}\right)
= \mathrm{vec}(\mathbf{e}_\ell\,\mathbf{a}_\ell^{\mathsf T})
= \mathbf{a}_\ell \otimes \mathbf{e}_\ell.
\]

This gives the key structural form used by KFAC.

\paragraph{Fisher block for a single
layer}\label{fisher-block-for-a-single-layer}

The Fisher block associated with the parameters \(W_\ell\) is \[
\mathcal{F}_\ell
= \mathbb{E}_{p(\mathbf{X})}\!\left[
\frac{\partial \log p(X\mid\theta)}{\partial \mathrm{vec}(W_\ell)}
\frac{\partial \log p(X\mid\theta)}{\partial \mathrm{vec}(W_\ell)}^{\mathsf T}
\right].
\]

Plugging in the expression above, \[
\mathcal{F}_\ell
= \mathbb{E}_{p(\mathbf{X})}\!\big[
(\mathbf{a}_\ell \otimes \mathbf{e}_\ell)
(\mathbf{a}_\ell \otimes \mathbf{e}_\ell)^{\mathsf T}
\big].
\]

Here \(p(\mathbf{X})\) denotes the distribution over inputs and labels
(or configurations, in the VMC case). In practice this expectation is
approximated by averaging over a mini-batch of samples \(X\) and the
corresponding forward/backward passes that produce \(\mathbf{a}_\ell\)
and \(\mathbf{e}_\ell\).

Computing and inverting \(\mathcal{F}_\ell\) directly is still
expensive, because its dimension is \[
(\text{dim}(\mathbf{a}_\ell)\,\text{dim}(\mathbf{e}_\ell))
\times
(\text{dim}(\mathbf{a}_\ell)\,\text{dim}(\mathbf{e}_\ell)).
\] KFAC makes two key approximations to make this tractable.

\begin{enumerate}
\def\labelenumi{\arabic{enumi}.}
\item
  \textbf{Block--diagonal across layers.}\\
  Off--diagonal blocks \(\mathcal{F}_{ij}\) are assumed negligible when
  \(\theta_i\) and \(\theta_j\) belong to different layers. This makes
  the Fisher approximately block--diagonal, with one block per layer.
\item
  \textbf{Kronecker factorization within each layer.}\\
  Inside a layer, KFAC assumes that the correlation between activations
  and errors factorizes: \[
  \mathcal{F}_\ell
  = \mathbb{E}_{p(\mathbf{X})}\!\big[
  (\mathbf{a}_\ell \otimes \mathbf{e}_\ell)
  (\mathbf{a}_\ell \otimes \mathbf{e}_\ell)^{\mathsf T}
  \big]
  = \mathbb{E}_{p(\mathbf{X})}\!\big[
  (\mathbf{a}_\ell\mathbf{a}_\ell^{\mathsf T}) \otimes
  (\mathbf{e}_\ell\mathbf{e}_\ell^{\mathsf T})
  \big]
  \;\approx\;
  \mathbb{E}_{p(\mathbf{X})}[\mathbf{a}_\ell\mathbf{a}_\ell^{\mathsf T}]
  \;\otimes\;
  \mathbb{E}_{p(\mathbf{X})}[\mathbf{e}_\ell\mathbf{e}_\ell^{\mathsf T}].
  \]
\end{enumerate}

Define the \emph{activation covariance} and \emph{error covariance}: \[
A_\ell = \mathbb{E}_{p(\mathbf{X})}[\mathbf{a}_\ell\mathbf{a}_\ell^{\mathsf T}],
\qquad
S_\ell = \mathbb{E}_{p(\mathbf{X})}[\mathbf{e}_\ell\mathbf{e}_\ell^{\mathsf T}].
\] In practice these expectations are updated as running averages over
mini-batches: \[
A_\ell \approx \frac{1}{B}\sum_{b=1}^B \mathbf{a}_\ell^{(b)} \mathbf{a}_\ell^{(b)\mathsf T},
\qquad
S_\ell \approx \frac{1}{B}\sum_{b=1}^B \mathbf{e}_\ell^{(b)} \mathbf{e}_\ell^{(b)\mathsf T},
\] where \(b\) indexes samples in the batch and
\(\mathbf{a}_\ell^{(b)}, \mathbf{e}_\ell^{(b)}\) are obtained by a
standard forward and backward pass for that sample. With this
approximation we have \[
\mathcal{F}_\ell \approx A_\ell \otimes S_\ell.
\]

The crucial property of the Kronecker product is that \[
(A_\ell \otimes S_\ell)^{-1}
= A_\ell^{-1} \otimes S_\ell^{-1},
\] so the inverse of the (huge) layer--Fisher block can be obtained by
inverting the much smaller matrices \(A_\ell\) and \(S_\ell\). Thus the
natural gradient update for the weights of layer \(\ell\) becomes \[
\Delta\theta_{\text{nat},\ell}
\approx -\,\eta\,\big(A_\ell^{-1} \otimes S_\ell^{-1}\big)\,
\nabla_{\mathrm{vec}(W_\ell)} \mathcal{L}.
\]

In summary, KFAC replaces the intractable inverse \[
\mathbb{E}_{p(\mathbf{X})}\big[ (\mathbf{a}_\ell\otimes \mathbf{e}_\ell)
(\mathbf{a}_\ell\otimes \mathbf{e}_\ell)^{\mathsf T} \big]^{-1}
\] by the efficiently computable approximation \[
\mathbb{E}_{p(\mathbf{X})}\big[(\mathbf{a}_\ell\otimes \mathbf{e}_\ell)
(\mathbf{a}_\ell\otimes \mathbf{e}_\ell)^{\mathsf T}\big]^{-1}
\;\approx\;
\mathbb{E}_{p(\mathbf{X})}[\mathbf{a}_\ell\mathbf{a}_\ell^{\mathsf T}]^{-1}
\otimes
\mathbb{E}_{p(\mathbf{X})}[\mathbf{e}_\ell\mathbf{e}_\ell^{\mathsf T}]^{-1},
\] which captures the dominant curvature structure while keeping the
cost of natural gradient descent comparable to standard first--order
methods. We have ignored biases above for clarity. In practice one can
either (i) augment \(\mathbf{a}_\ell\) with a constant \(1\) to absorb
biases into \(W_\ell\), or (ii) maintain separate smaller KFAC factors
for biases; both approaches preserve the same Kronecker structure.
\#\#\# Attention Mechanisms

The idea of an \emph{attention mechanism} was introduced in neural
machine translation by Bahdanau et al. @bahdanau2014neural Instead of
compressing an entire input sequence into a single fixed-size vector,
the model learns to \textbf{focus} on different parts of the input when
generating each output token.

Given a query vector \(\mathbf{q} \in \mathbb{R}^{d_h}\) and a set of
key--value pairs \(\{(\mathbf{k}_j, \mathbf{v}_j)\}_{j=1}^T\) with
\(\mathbf{k}_j, \mathbf{v}_j \in \mathbb{R}^{d_h}\), the (scaled
dot--product) attention mechanism computes:

\begin{enumerate}
\def\labelenumi{\arabic{enumi}.}
\item
  \textbf{Compatibility scores} \[
  e_j \;=\; \frac{\mathbf{q}^\top \mathbf{k}_j}{\sqrt{d_h}}, \qquad j = 1,\dots,T,
  \]
\item
  \textbf{Normalized attention weights} \[
  \alpha_j \;=\; \frac{\exp(e_j)}{\sum_{m=1}^{T} \exp(e_m)}
  \;=\; \text{Softmax}_j\!\left( \frac{\mathbf{q}^\top \mathbf{k}_j}{\sqrt{d_h}} \right),
  \]
\item
  \textbf{Weighted sum of values} \[
  \mathbf{o} \;=\; \sum_{j=1}^{T} \alpha_j \mathbf{v}_j.
  \] Intuitively, the query \(\mathbf{q}\) asks: \emph{``which elements
  of the set are relevant to me now?''}\\
  The keys \(\mathbf{k}_j\) encode \emph{what each element offers}, and
  the values \(\mathbf{v}_j\) encode \emph{what we take from each
  element once we decide to pay attention to it}.
\end{enumerate}

\subsubsection{Self-Attention and Multi-Head
Self-Attention}\label{self-attention-and-multi-head-self-attention}

In \textbf{self-attention}, the queries, keys, and values are all
obtained from the \textbf{same} set of input vectors.\\
Consider a sequence of input embeddings \[
\mathbf{x}_1, \dots, \mathbf{x}_T \in \mathbb{R}^{d},
\] and stack them into a matrix \[
\mathbf{X} \in \mathbb{R}^{T \times d}, \quad
\mathbf{X} = 
\begin{bmatrix}
\mathbf{x}_1^\top \\
\vdots \\
\mathbf{x}_T^\top
\end{bmatrix}.
\] To build one attention \textbf{head} of dimension \(d_h\), we
introduce three learnable matrices: \[
\mathbf{W}^Q \in \mathbb{R}^{d \times d_h}, \quad
\mathbf{W}^K \in \mathbb{R}^{d \times d_h}, \quad
\mathbf{W}^V \in \mathbb{R}^{d \times d_h}.
\] We then compute queries, keys, and values: \[
\mathbf{Q} = \mathbf{X} \mathbf{W}^Q \in \mathbb{R}^{T \times d_h}, \qquad
\mathbf{K} = \mathbf{X} \mathbf{W}^K \in \mathbb{R}^{T \times d_h}, \qquad
\mathbf{V} = \mathbf{X} \mathbf{W}^V \in \mathbb{R}^{T \times d_h}.
\]

The \textbf{scaled dot-product self-attention} for this head is: \[
\text{Attention}(\mathbf{Q}, \mathbf{K}, \mathbf{V})
= \text{Softmax}\!\left( \frac{\mathbf{Q}\mathbf{K}^\top}{\sqrt{d_h}} \right) \mathbf{V},
\] where the softmax is applied row-wise. Element-wise, the output at
position \(t\) is \[
\mathbf{o}_t = \sum_{j=1}^{T} \alpha_{tj} \mathbf{v}_j, \quad \text{with} \quad
\alpha_{tj} = \frac{\exp\!\left( \mathbf{q}_t^\top \mathbf{k}_j / \sqrt{d_h} \right)}{\sum_{m=1}^{T} \exp\!\left( \mathbf{q}_t^\top \mathbf{k}_m / \sqrt{d_h} \right)}.
\] You can think of this as: \emph{each position \(t\) in the sequence
``looks at'' every other position \(j\) and decides how much to care
about it}. \#\#\# Multi-Head Self-Attention

A single head can only look at interactions in one ``representation
subspace'' of dimension \(d_h\).\\
\textbf{Multi-head attention} uses several heads in parallel, each with
its own projection matrices, so that different types of relationships
can be captured simultaneously. Let \(n_h\) be the number of heads. For
head \(i = 1, \dots, n_h\) we have \[
\mathbf{W}_i^Q,\, \mathbf{W}_i^K,\, \mathbf{W}_i^V \in \mathbb{R}^{d \times d_h}.
\]

Head \(i\) computes: \[
\text{head}_i(\mathbf{X})
= \text{Attention}(\mathbf{X}\mathbf{W}_i^Q,\, \mathbf{X}\mathbf{W}_i^K,\, \mathbf{X}\mathbf{W}_i^V)
\in \mathbb{R}^{T \times d_h}.
\] The outputs of all heads are concatenated along the feature dimension
and then linearly mixed: \[
\mathbf{U} 
= \left[ \text{head}_1(\mathbf{X}) \,;\, \dots \,;\, \text{head}_{n_h}(\mathbf{X}) \right]
\in \mathbb{R}^{T \times (n_h d_h)},
\] \[
\mathbf{O} = \mathbf{U} \mathbf{W}^O, \qquad
\mathbf{W}^O \in \mathbb{R}^{(n_h d_h) \times d}.
\] If we focus on one time step \(t\) and head \(i\), we can write the
per-head output as \[
\mathbf{o}_{t,i} = \sum_{j=1}^{T} 
\text{Softmax}_j\!\left( \frac{\mathbf{q}_{t,i}^\top \mathbf{k}_{j,i}}{\sqrt{d_h}} \right)\mathbf{v}_{j,i},
\] and the final vector at time \(t\) after concatenation and output
projection as \[
\mathbf{u}_t =
\mathbf{W}^{O} 
\begin{bmatrix}
\mathbf{o}_{t,1} \\
\vdots \\
\mathbf{o}_{t,n_h}
\end{bmatrix}.
\] From a physics point of view, you can read multi-head attention as
\textbf{several different ``channels'' of interaction}: one head might
focus on short-range relations, another on long-range ones, another on
some specific pattern (e.g.~symmetry, local structure), and so on.

\subsubsection{Transformer Architecture}\label{transformer-architecture}

The \textbf{Transformer} was introduced by Vaswani et al. @Vaswani2017
with the slogan \emph{``Attention Is All You Need.''} Its core building
block is a \textbf{layer} that combines: 1. \textbf{Multi-head
self-attention}\\
2. \textbf{Position-wise feed-forward network (FFN)} Both sublayers use
\textbf{residual connections} and \textbf{layer normalization}. For an
input sequence \(\mathbf{X} \in \mathbb{R}^{T \times d}\) (already
including positional information), one Transformer layer performs: 3.
\textbf{Multi-head self-attention sublayer} \[
   \mathbf{H} = \text{MHA}(\mathbf{X}), \qquad
   \mathbf{X}^{(1)} = \text{LayerNorm}\!\left( \mathbf{X} + \mathbf{H} \right).
\] 4. \textbf{Feed-forward sublayer} (applied independently at each
position) \[
   \text{FFN}(\mathbf{x}) = \sigma\!\left( \mathbf{x}\mathbf{W}_1 + \mathbf{b}_1 \right)\mathbf{W}_2 + \mathbf{b}_2,
\] typically with \(\sigma\) a nonlinearity such as ReLU or GELU and an
intermediate width \(d_{\text{ff}} > d\). At the sequence level:\\
\[
   \mathbf{Z} = \text{FFN}(\mathbf{X}^{(1)}), \qquad
   \mathbf{X}^{\text{out}} = \text{LayerNorm}\!\left( \mathbf{X}^{(1)} + \mathbf{Z} \right).
\] Stacking several such layers yields a deep architecture where, at
each layer, every position can interact with every other position
through self-attention.

In the original formulation, \textbf{positional encodings} (sinusoidal
or learned) are added to the embeddings so that the model can
distinguish different positions in the sequence: \[
\mathbf{X}_0 = \mathbf{E} + \mathbf{P},
\] where \(\mathbf{E}\) are token embeddings and \(\mathbf{P}\) are
positional encodings.

\subsubsection{Why Transformers Instead of RNNs or
LSTMs?}\label{why-transformers-instead-of-rnns-or-lstms}

Recurrent Neural Networks (RNNs) and Long Short-Term Memory (LSTM)
networks process the sequence \textbf{sequentially} this is each new
state depends on the previous one. This has two important consequences:
1. Information must flow through many time steps, which can lead to
vanishing or exploding gradients and makes it hard to model very
long-range interactions. 2. Poor parallelization because each step
depends on the previous one, you cannot compute all time steps in
parallel. Training and inference are inherently sequential. Transformers
address both issues: - Global interactions in one step, self-attention
allows every position to directly interact with every other position in
a \emph{single} layer, which is ideal when we care about
\emph{all-to-all} correlations (as in many-electron systems, where each
electron ``feels'' all the others). - Full parallelism over sequence
length. Given \(\mathbf{X}\), the matrices \(\mathbf{Q}\),
\(\mathbf{K}\), \(\mathbf{V}\) and the attention outputs for all time
steps are computed via matrix multiplications. This is extremely
efficient on modern accelerators (GPUs/TPUs). For a many-electron
Schrödinger equation, the wave function depends on the joint
configuration of all particles. A Transformer-based ansatz naturally
provides a way for each electron's representation to \textbf{look at all
other electrons} and the nuclei, capturing complex correlation patterns
through attention, while remaining highly parallelizable.

A very important work for us is FermiNet (Pfau et al.~2020). It uses
deep neural networks to represent \textbf{orbitals} and then combines
them into a sum of Slater determinants. At the top level, the ansatz is
a linear combination of \(K\) determinant products \[
\psi(\mathbf{x}_1,\dots,\mathbf{x}_n)
= \sum_{k=1}^K \omega_k \,\det[\Phi^{k}],
\] where \(\omega_k\) are learnable coefficients and \(\Phi^k\) is a
matrix of single-particle orbitals. For a system without explicit spin
separation one can write \[
\det[\Phi^k] =
\begin{vmatrix}
\phi_{1}^{k}(\mathbf{x}_{1})  & \dots  &  \phi_{1}^{k}(\mathbf{x}_{n}) \\
\vdots   &  & \vdots  \\
\phi_{n}^{k}(\mathbf{x}_{1}) & \dots & \phi_{n}^{k}(\mathbf{x}_{n})
\end{vmatrix}
= \det[\phi_i^k(\mathbf{x}_j)].
\] Here \(\phi_i^k\) is the \(i\)-th orbital in determinant \(k\), and
we evaluate it on the coordinates of electron \(j\).

However, in FermiNet we are dealing with electrons with spin, so things
are slightly more structured, and the orbitals depend on \textbf{all}
electron coordinates, not only on the one being ``plugged in''. That is
why we write the orbitals as \[
\phi^{k\alpha}_i\big(\mathbf{r}^\alpha_j;\{\mathbf{r}^\alpha_{/j}\};\{\mathbf{r}^{\bar{\alpha}}\}\big),
\] where: - \(\alpha \in \{\uparrow,\downarrow\}\) is the spin sector, -
\(\mathbf{r}^\alpha_j\) is the position of electron \(j\) with spin
\(\alpha\), - \(\{\mathbf{r}^\alpha_{/j}\}\) denotes the positions of
all \textbf{other} electrons with spin \(\alpha\), -
\(\{\mathbf{r}^{\bar{\alpha}}\}\) denotes the positions of electrons
with the opposite spin.

So the orbital evaluated on electron \(j\) ``knows'' about all other
electrons. The indices: - \(i\) = orbital index (row of the
determinant), - \(j\) = electron index (column of the determinant), -
\(\alpha,\beta\) = spin labels (\(\uparrow\) or \(\downarrow\)), - \(k\)
= determinant index in the sum.

\begin{center}\rule{0.5\linewidth}{0.5pt}\end{center}

\subsubsection{Input coordinates and
features}\label{input-coordinates-and-features}

We denote by -
\(\mathbf{r}^\uparrow_1,\dots,\mathbf{r}^\uparrow_{n^\uparrow}\) the
coordinates of spin-up electrons, -
\(\mathbf{r}^\downarrow_1,\dots,\mathbf{r}^\downarrow_{n^\downarrow}\)
the coordinates of spin-down electrons, - \(\mathbf{R}_I\) the positions
of nuclei, \(I=1,\dots,N_\text{nuc}\).

The network builds two types of features:

\begin{enumerate}
\def\labelenumi{\arabic{enumi}.}
\item
  \textbf{Electron--nucleus features} for each electron \(i\) with spin
  \(\alpha\): \[
  \mathbf{h}^{0,\alpha}_i
  = \text{concatenate}\Big(
      \mathbf{r}^\alpha_i - \mathbf{R}_I,\;
      \big|\mathbf{r}^\alpha_i - \mathbf{R}_I\big|
      \ \forall\, I
    \Big).
  \] This produces a feature vector that contains, for electron
  \((i,\alpha)\), all its relative position vectors to each nucleus,
  plus their distances.
\item
  \textbf{Electron--electron features} for each pair of electrons
  \((i,\alpha)\) and \((j,\beta)\): \[
  \mathbf{h}^{0,\alpha\beta}_{ij}
  = \text{concatenate}\Big(
      \mathbf{r}^\alpha_i - \mathbf{r}^\beta_j,\;
      \big|\mathbf{r}^\alpha_i - \mathbf{r}^\beta_j\big|
      \ \forall\, j,\beta
    \Big).
  \] For fixed \((i,\alpha)\), we build such features for all other
  electrons \((j,\beta)\), capturing their relative positions and
  distances.
\end{enumerate}

The superscript \(0\) indicates that these are the features at layer
\(\ell=0\) (input to the deep network). At deeper layers we will keep
updating - \(\mathbf{h}^{\ell\alpha}_i\) (single-electron features), -
\(\mathbf{h}^{\ell\alpha\beta}_{ij}\) (pairwise features), for
\(\ell = 0,1,\dots,L-1\).

\begin{center}\rule{0.5\linewidth}{0.5pt}\end{center}

\subsubsection{Mixing and updating features across
layers}\label{mixing-and-updating-features-across-layers}

At each hidden layer \(\ell\), we want each electron's features to
depend on \emph{all} other electrons, in a permutation-symmetric way. To
do this, we form \textbf{averages} over electrons of the same or
opposite spin.

First, define global spin-averaged single-electron features \[
\mathbf{g}^{\ell\uparrow} =
\frac{1}{n^\uparrow}\sum_{j=1}^{n^\uparrow}\mathbf{h}^{\ell\uparrow}_j,
\qquad
\mathbf{g}^{\ell\downarrow} =
\frac{1}{n^\downarrow}\sum_{j=1}^{n^\downarrow}\mathbf{h}^{\ell\downarrow}_j.
\]

Next, for each electron \((i,\alpha)\), define averaged pairwise
features: \[
\mathbf{g}^{\ell\alpha\uparrow}_i
= \frac{1}{n^\uparrow}\sum_{j=1}^{n^\uparrow}\mathbf{h}^{\ell\alpha\uparrow}_{ij},
\qquad
\mathbf{g}^{\ell\alpha\downarrow}_i
= \frac{1}{n^\downarrow}\sum_{j=1}^{n^\downarrow}\mathbf{h}^{\ell\alpha\downarrow}_{ij}.
\]

Now we \emph{concatenate} all this information into a single feature
vector for electron \((i,\alpha)\): \[
\begin{aligned}
\big(
\mathbf{h}^{\ell\alpha}_i,
\frac{1}{n^\uparrow}\sum_{j=1}^{n^\uparrow}\mathbf{h}^{\ell\uparrow}_j,
\frac{1}{n^\downarrow}\sum_{j=1}^{n^\downarrow}\mathbf{h}^{\ell\downarrow}_j,
\frac{1}{n^\uparrow}\sum_{j=1}^{n^\uparrow}\mathbf{h}^{\ell\alpha\uparrow}_{ij},
\frac{1}{n^\downarrow}\sum_{j=1}^{n^\downarrow}\mathbf{h}^{\ell\alpha\downarrow}_{ij}
\big)
&=
\big(\mathbf{h}^{\ell\alpha}_i, \mathbf{g}^{\ell\uparrow}, \mathbf{g}^{\ell\downarrow},
\mathbf{g}^{\ell\alpha\uparrow}_i, \mathbf{g}^{\ell\alpha\downarrow}_i \big) \\
&= \mathbf{f}^{\ell\alpha}_i.
\end{aligned}
\]

This \(\mathbf{f}^{\ell\alpha}_i\) is what enters the
\textbf{single-electron MLP} at layer \(\ell\). The update is \[
\mathbf{h}^{\ell+1,\alpha}_i
= \tanh\big(\mathbf{V}^\ell \mathbf{f}^{\ell\alpha}_i + \mathbf{b}^\ell\big) + \mathbf{h}^{\ell\alpha}_i,
\] where \(\mathbf{V}^\ell\) and \(\mathbf{b}^\ell\) are learnable
weights and biases, shared between electrons (for the given spin
sector). The residual connection \(+\mathbf{h}^{\ell\alpha}_i\)
stabilizes training.

In parallel, the pairwise features are updated with a \textbf{pairwise
MLP}: \[
\mathbf{h}^{\ell+1,\alpha\beta}_{ij}
= \tanh\big(\mathbf{W}^\ell \mathbf{h}^{\ell\alpha\beta}_{ij} + \mathbf{c}^\ell\big)
+ \mathbf{h}^{\ell\alpha\beta}_{ij},
\] with weights \(\mathbf{W}^\ell\) and biases \(\mathbf{c}^\ell\),
again shared over all pairs \((i,j,\alpha,\beta)\).

By repeating these updates for \(\ell = 0,\dots,L-1\), we eventually
obtain \textbf{final single-electron features} \[
\mathbf{h}^{L\alpha}_j \quad \text{for each electron } j \text{ of spin } \alpha.
\] Notice how the indices work: - \(\ell\) runs over layers and
disappears at the end, - \(i\) or \(j\) always refer to a specific
electron within a spin sector, - \(\alpha,\beta\) tell you which spin
sector that electron belongs to.

\begin{center}\rule{0.5\linewidth}{0.5pt}\end{center}

\subsubsection{From final features to
orbitals}\label{from-final-features-to-orbitals}

The final orbitals are built as a function of the last-layer features
\(\mathbf{h}^{L\alpha}_j\), plus some additional ``envelope'' factors
that handle the long-range decay and cusp conditions. For each
determinant index \(k\), spin \(\alpha\), orbital index \(i\), and
electron \(j\) we define \[
\begin{aligned}
\phi^{k\alpha}_i\big(\mathbf{r}^\alpha_j; \{\mathbf{r}^\alpha_{/j}\}; \{\mathbf{r}^{\bar{\alpha}}\}\big)
&= \left(\mathbf{w}^{k\alpha}_i \cdot \mathbf{h}^{L\alpha}_j + g^{k\alpha}_i\right) \\
&\quad\times \sum_{m} \pi^{k\alpha}_{im}
\exp\Big(
- \big|\mathbf{\Sigma}_{im}^{k\alpha} \big(\mathbf{r}^{\alpha}_j - \mathbf{R}_m\big)\big|
\Big).
\end{aligned}
\] Here: - \(\mathbf{w}^{k\alpha}_i\) and \(g^{k\alpha}_i\) are
learnable linear parameters for the ``MLP part'' of the orbital, - the
sum over \(m\) is an ``envelope'' over nuclei (or centers), -
\(\pi^{k\alpha}_{im}\) and \(\mathbf{\Sigma}^{k\alpha}_{im}\) are
learnable coefficients and matrices controlling the exponential decay
around nucleus \(m\).

All these parameters depend on the indices: - \(k\) selects which
determinant in the sum, - \(i\) selects which orbital (row in the
determinant), - \(\alpha\) selects the spin sector, - \(m\) selects
which nuclear center in the envelope.

The dependence on all other electrons is hidden inside
\(\mathbf{h}^{L\alpha}_j\), which was built from the full set of
positions \(\{\mathbf{r}^\uparrow\},\{\mathbf{r}^\downarrow\}\) through
the deep network.

\begin{center}\rule{0.5\linewidth}{0.5pt}\end{center}

\subsubsection{Assembling the spin-separated
determinants}\label{assembling-the-spin-separated-determinants}

For each determinant index \(k\) and spin sector
\(\alpha\in\{\uparrow,\downarrow\}\), we build a matrix \[
D^{k\alpha}_{ij}
= \phi^{k\alpha}_i\big( \mathbf{r}^\alpha_j; \{\mathbf{r}^\alpha_{/j}\}; \{\mathbf{r}^{\bar{\alpha}}\}\big),
\] with: - rows indexed by the orbital label \(i = 1,\dots,n^\alpha\), -
columns indexed by the electron label \(j = 1,\dots,n^\alpha\) (with
that spin).

Taking the determinant gives a properly antisymmetric function of the
positions of electrons \textbf{with that spin}: \[
\det\big[D^{k\alpha}\big]
= \det\left[\phi^{k\alpha}_i(\mathbf{r}^\alpha_j; \{\mathbf{r}^\alpha_{/j}\}; \{\mathbf{r}^{\bar{\alpha}}\})\right].
\]

For the full wavefunction, we combine spin-up and spin-down blocks: \[
\begin{aligned}
\psi(\mathbf{r}^\uparrow_1,\ldots,\mathbf{r}^\uparrow_{n^\uparrow},
     \mathbf{r}^\downarrow_1,\ldots,\mathbf{r}^\downarrow_{n^\downarrow})
= \sum_{k} \omega_k \;&
\det\left[\phi^{k \uparrow}_i(\mathbf{r}^\uparrow_j; \{\mathbf{r}^\uparrow_{/j}\}; \{\mathbf{r}^\downarrow\})\right] \\
&\times
\det\left[\phi^{k \downarrow}_i(\mathbf{r}^\downarrow_j; \{\mathbf{r}^\downarrow_{/j}\}; \{\mathbf{r}^\uparrow\})\right].
\end{aligned}
\]

\textbf{Why are there two determinants?}

In electronic structure, when we separate spin and spatial parts using
spin-orbitals, the full Slater determinant over all electrons factorizes
into the product of: - one determinant involving only spin-up electrons,
- another determinant involving only spin-down electrons.

Each of these determinants is antisymmetric under exchange of two
electrons \textbf{with the same spin}. The overall wavefunction
constructed as the product of a spin-up determinant and a spin-down
determinant is antisymmetric under exchange of any two electrons (when
you take into account the spin labels). FermiNet keeps this structure
and lets each block be represented by a powerful neural network ansatz
for the orbitals.

Up to this point the building blocks are just MLP layers (with residual
connections and special feature mixing), but the careful indexing -
\((i,\alpha)\) for ``which electron/spin'', - \(j\) for summation over
electrons, - \(\ell\) for layers, - \(k\) for determinant index, is what
guarantees that the final object has the correct permutation symmetry
and antisymmetry required for a fermionic wavefunction.

\section{Psi Former Model}\label{psi-former-model}

\subsection{Fermi Net}\label{fermi-net}

A very important work for us is FermiNet (Pfau et al.~2020). It uses
deep neural networks to represent \textbf{orbitals} and then combines
them into a sum of Slater determinants. At the top level, the ansatz is
a linear combination of \(K\) determinant products \[
\psi(\mathbf{x}_1,\dots,\mathbf{x}_n)
= \sum_{k=1}^K \omega_k \,\det[\Phi^{k}],
\] where \(\omega_k\) are learnable coefficients and \(\Phi^k\) is a
matrix of single-particle orbitals. For a system without explicit spin
separation one can write \[
\det[\Phi^k] =
\begin{vmatrix}
\phi_{1}^{k}(\mathbf{x}_{1})  & \dots  &  \phi_{1}^{k}(\mathbf{x}_{n}) \\
\vdots   &  & \vdots  \\
\phi_{n}^{k}(\mathbf{x}_{1}) & \dots & \phi_{n}^{k}(\mathbf{x}_{n})
\end{vmatrix}
= \det[\phi_i^k(\mathbf{x}_j)].
\] Here \(\phi_i^k\) is the \(i\)-th orbital in determinant \(k\), and
we evaluate it on the coordinates of electron \(j\).

However, in FermiNet we are dealing with electrons with spin, so things
are slightly more structured, and the orbitals depend on \textbf{all}
electron coordinates, not only on the one being ``plugged in''. That is
why we write the orbitals as \[
\phi^{k\alpha}_i\big(\mathbf{r}^\alpha_j;\{\mathbf{r}^\alpha_{/j}\};\{\mathbf{r}^{\bar{\alpha}}\}\big),
\] where: - \(\alpha \in \{\uparrow,\downarrow\}\) is the spin sector, -
\(\mathbf{r}^\alpha_j\) is the position of electron \(j\) with spin
\(\alpha\), - \(\{\mathbf{r}^\alpha_{/j}\}\) denotes the positions of
all \textbf{other} electrons with spin \(\alpha\), -
\(\{\mathbf{r}^{\bar{\alpha}}\}\) denotes the positions of electrons
with the opposite spin.

So the orbital evaluated on electron \(j\) ``knows'' about all other
electrons. The indices: - \(i\) = orbital index (row of the
determinant), - \(j\) = electron index (column of the determinant), -
\(\alpha,\beta\) = spin labels (\(\uparrow\) or \(\downarrow\)), - \(k\)
= determinant index in the sum.

\begin{center}\rule{0.5\linewidth}{0.5pt}\end{center}

\subsubsection{Input coordinates and
features}\label{input-coordinates-and-features-1}

We denote by -
\(\mathbf{r}^\uparrow_1,\dots,\mathbf{r}^\uparrow_{n^\uparrow}\) the
coordinates of spin-up electrons, -
\(\mathbf{r}^\downarrow_1,\dots,\mathbf{r}^\downarrow_{n^\downarrow}\)
the coordinates of spin-down electrons, - \(\mathbf{R}_I\) the positions
of nuclei, \(I=1,\dots,N_\text{nuc}\).

The network builds two types of features:

\begin{enumerate}
\def\labelenumi{\arabic{enumi}.}
\item
  \textbf{Electron--nucleus features} for each electron \(i\) with spin
  \(\alpha\): \[
  \mathbf{h}^{0,\alpha}_i
  = \text{concatenate}\Big(
      \mathbf{r}^\alpha_i - \mathbf{R}_I,\;
      \big|\mathbf{r}^\alpha_i - \mathbf{R}_I\big|
      \ \forall\, I
    \Big).
  \] This produces a feature vector that contains, for electron
  \((i,\alpha)\), all its relative position vectors to each nucleus,
  plus their distances.
\item
  \textbf{Electron--electron features} for each pair of electrons
  \((i,\alpha)\) and \((j,\beta)\): \[
  \mathbf{h}^{0,\alpha\beta}_{ij}
  = \text{concatenate}\Big(
      \mathbf{r}^\alpha_i - \mathbf{r}^\beta_j,\;
      \big|\mathbf{r}^\alpha_i - \mathbf{r}^\beta_j\big|
      \ \forall\, j,\beta
    \Big).
  \] For fixed \((i,\alpha)\), we build such features for all other
  electrons \((j,\beta)\), capturing their relative positions and
  distances.
\end{enumerate}

The superscript \(0\) indicates that these are the features at layer
\(\ell=0\) (input to the deep network). At deeper layers we will keep
updating - \(\mathbf{h}^{\ell\alpha}_i\) (single-electron features), -
\(\mathbf{h}^{\ell\alpha\beta}_{ij}\) (pairwise features), for
\(\ell = 0,1,\dots,L-1\).

\begin{center}\rule{0.5\linewidth}{0.5pt}\end{center}

\subsubsection{Mixing and updating features across
layers}\label{mixing-and-updating-features-across-layers-1}

At each hidden layer \(\ell\), we want each electron's features to
depend on \emph{all} other electrons, in a permutation-symmetric way. To
do this, we form \textbf{averages} over electrons of the same or
opposite spin.

First, define global spin-averaged single-electron features \[
\mathbf{g}^{\ell\uparrow} =
\frac{1}{n^\uparrow}\sum_{j=1}^{n^\uparrow}\mathbf{h}^{\ell\uparrow}_j,
\qquad
\mathbf{g}^{\ell\downarrow} =
\frac{1}{n^\downarrow}\sum_{j=1}^{n^\downarrow}\mathbf{h}^{\ell\downarrow}_j.
\]

Next, for each electron \((i,\alpha)\), define averaged pairwise
features: \[
\mathbf{g}^{\ell\alpha\uparrow}_i
= \frac{1}{n^\uparrow}\sum_{j=1}^{n^\uparrow}\mathbf{h}^{\ell\alpha\uparrow}_{ij},
\qquad
\mathbf{g}^{\ell\alpha\downarrow}_i
= \frac{1}{n^\downarrow}\sum_{j=1}^{n^\downarrow}\mathbf{h}^{\ell\alpha\downarrow}_{ij}.
\]

Now we \emph{concatenate} all this information into a single feature
vector for electron \((i,\alpha)\): \[
\begin{aligned}
\big(
\mathbf{h}^{\ell\alpha}_i,
\frac{1}{n^\uparrow}\sum_{j=1}^{n^\uparrow}\mathbf{h}^{\ell\uparrow}_j,
\frac{1}{n^\downarrow}\sum_{j=1}^{n^\downarrow}\mathbf{h}^{\ell\downarrow}_j,
\frac{1}{n^\uparrow}\sum_{j=1}^{n^\uparrow}\mathbf{h}^{\ell\alpha\uparrow}_{ij},
\frac{1}{n^\downarrow}\sum_{j=1}^{n^\downarrow}\mathbf{h}^{\ell\alpha\downarrow}_{ij}
\big)
&=
\big(\mathbf{h}^{\ell\alpha}_i, \mathbf{g}^{\ell\uparrow}, \mathbf{g}^{\ell\downarrow},
\mathbf{g}^{\ell\alpha\uparrow}_i, \mathbf{g}^{\ell\alpha\downarrow}_i \big) \\
&= \mathbf{f}^{\ell\alpha}_i.
\end{aligned}
\]

This \(\mathbf{f}^{\ell\alpha}_i\) is what enters the
\textbf{single-electron MLP} at layer \(\ell\). The update is \[
\mathbf{h}^{\ell+1,\alpha}_i
= \tanh\big(\mathbf{V}^\ell \mathbf{f}^{\ell\alpha}_i + \mathbf{b}^\ell\big) + \mathbf{h}^{\ell\alpha}_i,
\] where \(\mathbf{V}^\ell\) and \(\mathbf{b}^\ell\) are learnable
weights and biases, shared between electrons (for the given spin
sector). The residual connection \(+\mathbf{h}^{\ell\alpha}_i\)
stabilizes training.

In parallel, the pairwise features are updated with a \textbf{pairwise
MLP}: \[
\mathbf{h}^{\ell+1,\alpha\beta}_{ij}
= \tanh\big(\mathbf{W}^\ell \mathbf{h}^{\ell\alpha\beta}_{ij} + \mathbf{c}^\ell\big)
+ \mathbf{h}^{\ell\alpha\beta}_{ij},
\] with weights \(\mathbf{W}^\ell\) and biases \(\mathbf{c}^\ell\),
again shared over all pairs \((i,j,\alpha,\beta)\).

By repeating these updates for \(\ell = 0,\dots,L-1\), we eventually
obtain \textbf{final single-electron features} \[
\mathbf{h}^{L\alpha}_j \quad \text{for each electron } j \text{ of spin } \alpha.
\] Notice how the indices work: - \(\ell\) runs over layers and
disappears at the end, - \(i\) or \(j\) always refer to a specific
electron within a spin sector, - \(\alpha,\beta\) tell you which spin
sector that electron belongs to.

\begin{center}\rule{0.5\linewidth}{0.5pt}\end{center}

\subsubsection{From final features to
orbitals}\label{from-final-features-to-orbitals-1}

The final orbitals are built as a function of the last-layer features
\(\mathbf{h}^{L\alpha}_j\), plus some additional ``envelope'' factors
that handle the long-range decay and cusp conditions. For each
determinant index \(k\), spin \(\alpha\), orbital index \(i\), and
electron \(j\) we define \[
\begin{aligned}
\phi^{k\alpha}_i\big(\mathbf{r}^\alpha_j; \{\mathbf{r}^\alpha_{/j}\}; \{\mathbf{r}^{\bar{\alpha}}\}\big)
&= \left(\mathbf{w}^{k\alpha}_i \cdot \mathbf{h}^{L\alpha}_j + g^{k\alpha}_i\right) \\
&\quad\times \sum_{m} \pi^{k\alpha}_{im}
\exp\Big(
- \big|\mathbf{\Sigma}_{im}^{k\alpha} \big(\mathbf{r}^{\alpha}_j - \mathbf{R}_m\big)\big|
\Big).
\end{aligned}
\] Here: - \(\mathbf{w}^{k\alpha}_i\) and \(g^{k\alpha}_i\) are
learnable linear parameters for the ``MLP part'' of the orbital, - the
sum over \(m\) is an ``envelope'' over nuclei (or centers), -
\(\pi^{k\alpha}_{im}\) and \(\mathbf{\Sigma}^{k\alpha}_{im}\) are
learnable coefficients and matrices controlling the exponential decay
around nucleus \(m\).

All these parameters depend on the indices: - \(k\) selects which
determinant in the sum, - \(i\) selects which orbital (row in the
determinant), - \(\alpha\) selects the spin sector, - \(m\) selects
which nuclear center in the envelope.

The dependence on all other electrons is hidden inside
\(\mathbf{h}^{L\alpha}_j\), which was built from the full set of
positions \(\{\mathbf{r}^\uparrow\},\{\mathbf{r}^\downarrow\}\) through
the deep network.

\begin{center}\rule{0.5\linewidth}{0.5pt}\end{center}

\subsubsection{Assembling the spin-separated
determinants}\label{assembling-the-spin-separated-determinants-1}

For each determinant index \(k\) and spin sector
\(\alpha\in\{\uparrow,\downarrow\}\), we build a matrix \[
D^{k\alpha}_{ij}
= \phi^{k\alpha}_i\big( \mathbf{r}^\alpha_j; \{\mathbf{r}^\alpha_{/j}\}; \{\mathbf{r}^{\bar{\alpha}}\}\big),
\] with: - rows indexed by the orbital label \(i = 1,\dots,n^\alpha\), -
columns indexed by the electron label \(j = 1,\dots,n^\alpha\) (with
that spin).

Taking the determinant gives a properly antisymmetric function of the
positions of electrons \textbf{with that spin}: \[
\det\big[D^{k\alpha}\big]
= \det\left[\phi^{k\alpha}_i(\mathbf{r}^\alpha_j; \{\mathbf{r}^\alpha_{/j}\}; \{\mathbf{r}^{\bar{\alpha}}\})\right].
\]

For the full wavefunction, we combine spin-up and spin-down blocks: \[
\begin{aligned}
\psi(\mathbf{r}^\uparrow_1,\ldots,\mathbf{r}^\uparrow_{n^\uparrow},
     \mathbf{r}^\downarrow_1,\ldots,\mathbf{r}^\downarrow_{n^\downarrow})
= \sum_{k} \omega_k \;&
\det\left[\phi^{k \uparrow}_i(\mathbf{r}^\uparrow_j; \{\mathbf{r}^\uparrow_{/j}\}; \{\mathbf{r}^\downarrow\})\right] \\
&\times
\det\left[\phi^{k \downarrow}_i(\mathbf{r}^\downarrow_j; \{\mathbf{r}^\downarrow_{/j}\}; \{\mathbf{r}^\uparrow\})\right].
\end{aligned}
\] We have don't explained why we can write the determinant as that
product. In electronic structure, when we separate spin and spatial
parts using spin-orbitals, the full Slater determinant over all
electrons factorizes into the product of: one determinant involving only
spin-up electrons, another determinant involving only spin-down
electrons.

Each of these determinants is antisymmetric under exchange of two
electrons \textbf{with the same spin}. The overall wavefunction
constructed as the product of a spin-up determinant and a spin-down
determinant is antisymmetric under exchange of any two electrons (when
you take into account the spin labels). FermiNet keeps this structure
and lets each block be represented by a powerful neural network ansatz
for the orbitals.

Up to this point the building blocks are just MLP layers (with residual
connections and special feature mixing), but the careful indexing -
\((i,\alpha)\) for ``which electron/spin'', - \(j\) for summation over
electrons, - \(\ell\) for layers, - \(k\) for determinant index, is what
guarantees that the final object has the correct permutation symmetry
and antisymmetry required for a fermionic wavefunction.

!{[}{[}ferminet.png\textbar280x315{]}{]}

(Architecture of Fermite Source )

\subsection{Jastrow Factor for Psi
Former}\label{jastrow-factor-for-psi-former}

{[}{[}Psi Former Ansatz{]}{]}.
@vonglehn2023selfattentionansatzabinitioquantum

The Psiformer wavefunction has the usual Slater--Jastrow structure \[
\Psi_{\theta}(\mathbf{x})
=
\exp\big(\mathcal{J}_{\theta}(\mathbf{x})\big)\,
\sum_{k=1}^{N_{\det}}\det[\boldsymbol{\Phi}^{k}_{\theta}(\mathbf{x})],
\] where \(\mathbf{x} = (x_1,\dots,x_N)\) is the collection of all \(N\)
electron states \[
x_i = (\mathbf{r}_i,\sigma_i), \qquad \mathbf{r}_i \in \mathbb{R}^3,\;\sigma_i \in \{\uparrow,\downarrow\}.
\]

\begin{itemize}
\tightlist
\item
  \(\mathcal{J}_\theta:(\mathbb{R}^{3}\times \{\uparrow,\downarrow\})^{N}\to \mathbb{R}\)
  is the \textbf{Jastrow factor}, encoding (here) only
  electron--electron cusp information.
\item
  \(\boldsymbol{\Phi}^k_\theta\) is the matrix of (spin-)orbitals for
  determinant \(k\).
\end{itemize}

In Psiformer, the Jastrow factor is \emph{very} simple: it has only two
learnable parameters, one for parallel-spin pairs and one for
antiparallel-spin pairs: \[
\mathcal{J}_{\theta}(\mathbf{x})
=
\sum_{i<j;\,\sigma_{i}=\sigma_{j}}
-\frac{1}{4}\frac{\alpha^{2}_{\mathrm{par}}}{\alpha_{\mathrm{par}}+\lvert \mathbf{r}_{i}-\mathbf{r}_{j} \rvert }
\;+\;
\sum_{i,j;\,\sigma_{i}\neq \sigma_{j}}
-\frac{1}{2}\frac{\alpha^{2}_{\mathrm{anti}}}{\alpha_{\mathrm{anti}}+\lvert \mathbf{r}_{i}-\mathbf{r}_{j} \rvert }.
\]

\begin{itemize}
\tightlist
\item
  \(\alpha_{\mathrm{par}}\) controls the strength of the Jastrow for
  \textbf{same-spin} electron pairs.
\item
  \(\alpha_{\mathrm{anti}}\) does the same for \textbf{opposite-spin}
  pairs.
\end{itemize}

This Jastrow is responsible for enforcing the electron--electron cusp
conditions. The neural network itself (the Psiformer) only sees
\textbf{electron--nucleus} information in its attention stream; all
explicit \(|\mathbf{r}_i-\mathbf{r}_j|\) dependence lives in
\(\mathcal{J}_\theta\).

\begin{center}\rule{0.5\linewidth}{0.5pt}\end{center}

\subsection{Applying Attention to Fermi Net
(Psiformer-style)}\label{applying-attention-to-fermi-net-psiformer-style}

Conceptually, Psiformer is ``FermiNet with the two-electron stream
replaced by self-attention'', we can see it clearly doing.

\begin{itemize}
\tightlist
\item
  FermiNet: separate one-electron and two-electron feature streams, then
  mix.
\item
  Psiformer: a \textbf{single stream} of self-attention layers on
  electron--nuclear features only; electron--electron features enter
  only via the Jastrow.
\end{itemize}

We now explain the indices and equations carefully. \#\#\# Indices

We will use:

\begin{itemize}
\tightlist
\item
  \(i,j = 1,\dots,N\): electron indices.
\item
  \(I = 1,\dots,N_{\text{nuc}}\): nucleus index.
\item
  \(\sigma_i \in \{\uparrow,\downarrow\}\): spin of electron \(i\).
\item
  \(\ell = 0,\dots,L-1\): layer index in the Psiformer.
\item
  \(h = 1,\dots,H\): attention head index.
\item
  \(k = 1,\dots,N_{\det}\): determinant index.
\item
  \(d\): hidden dimension of the per-electron feature vectors. So at
  each layer \(\ell\), each electron \(i\) carries a feature (hidden
  state) \[
  \mathbf{h}_i^{\ell} \in \mathbb{R}^{d}.
  \] \#\#\# Input features and initial hidden states
\end{itemize}

Psiformer only uses \textbf{electron--nuclear} features (plus spin) as
input to the attention stack. For each electron \(i\): 1. Let
\(\mathbf{R}_I\) be nuclear positions. 2. Build raw features by
concatenating for all \(I\): - some function of
\(\mathbf{r}_i - \mathbf{R}_I\) (relative position), -
\(|\mathbf{r}_i - \mathbf{R}_I|\) (distance), - and the spin as a scalar
(e.g.~\(\sigma_i = +1\) for \(\uparrow\), \(-1\) for \(\downarrow\)).

In the paper they rescale the electron--nucleus vectors so that large
distances grow only logarithmically, but at the level of notation we can
just write \[
\mathbf{f}_i^{0} \in \mathbb{R}^{d_{\text{in}}}
\quad\text{(electron–nucleus features + spin)}.
\] These are then mapped into the model hidden dimension by a linear
layer \[
\mathbf{h}_{i}^{0} = \mathbf{W}^{0}\,\mathbf{f}_{i}^{0},
\] where \(\mathbf{W}^0 \in \mathbb{R}^{d \times d_{\text{in}}}\) is
learned. So: - index \(i\) is ``which electron'', - superscript \(0\)
means ``before any attention layers.'' \#\#\# One self-attention layer

At layer \(\ell\), we have all electron hidden states \[
\mathbf{h}_1^{\ell},\dots,\mathbf{h}_N^{\ell}.
\]

For each \textbf{head} \(h\) and electron \(i\), we compute:

\begin{itemize}
\tightlist
\item
  Query: \[
  \mathbf{q}^{\ell h}_i = \mathbf{W}^{\ell h}_q \mathbf{h}^{\ell}_i
  \]
\item
  Key: \[
  \mathbf{k}^{\ell h}_i = \mathbf{W}^{\ell h}_k \mathbf{h}^{\ell}_i
  \]
\item
  Value: \[
  \mathbf{v}^{\ell h}_i = \mathbf{W}^{\ell h}_v \mathbf{h}^{\ell}_i
  \]
\end{itemize}

Here each
\(\mathbf{W}^{\ell h}_q,\mathbf{W}^{\ell h}_k,\mathbf{W}^{\ell h}_v\) is
a learned matrix, shared across all electrons \(i\), but specific to
layer \(\ell\) and head \(h\).

Then the \textbf{self-attention output for electron \(i\), head \(h\)}
is \[
\mathbf{A}^{\ell h}_i
=
\sum_{j=1}^{N}
\underbrace{
\frac{\exp\big((\mathbf{q}^{\ell h}_i)^{\mathsf T}\mathbf{k}^{\ell h}_j / \sqrt{d_k}\big)}
     {\sum_{j'=1}^N \exp\big((\mathbf{q}^{\ell h}_i)^{\mathsf T}\mathbf{k}^{\ell h}_{j'} / \sqrt{d_k}\big)}
}_{\text{attention weight from } i \text{ to } j}
\mathbf{v}^{\ell h}_j.
\]

\begin{itemize}
\tightlist
\item
  \(j\) runs over ``all other electrons,'' so electron \(i\) ``looks''
  at all others via attention.
\item
  \(d_k\) is the key/query dimension (usually \(d_k = d/H\) or similar).
\end{itemize}

This is exactly your \[
A^{\ell}_{h} = [\text{SelfAttn}(\mathbf{h}_1^\ell,\dots,\mathbf{h}_N^\ell;\mathbf{W}^{\ell h}_q,\mathbf{W}^{\ell h}_k,\mathbf{W}^{\ell h}_v)],
\] but now written explicitly with indices \(i\) and \(j\).

Next, we \textbf{concatenate over heads} for each electron: \[
\mathbf{A}^{\ell}_i = \text{concat}_{h=1}^H\big[\mathbf{A}^{\ell h}_i\big]
\in \mathbb{R}^{Hd_v},
\] where \(d_v\) is the value dimension of each head.

This is your \[
A^{\ell} = \text{concat}_{h}[A_{h}],
\] but again with the electron index \(i\) made explicit.

\subsubsection{Residual projection and
MLP}\label{residual-projection-and-mlp}

We then map the concatenated attention output back to the hidden
dimension and add a residual connection: \[
\mathbf{f}_{i}^{\ell+1}
=
\mathbf{h}_{i}^{\ell}
+
\mathbf{W}_{o}^{\ell}\,\mathbf{A}^{\ell}_i,
\] where \(\mathbf{W}_{o}^{\ell}\) is a learned matrix.

Then we pass this through a small MLP (just a linear + \(\tanh\) here),
again with a residual: \[
\mathbf{h}_{i}^{\ell+1}
=
\mathbf{f}_{i}^{\ell+1}
+
\tanh\big(\mathbf{W}^{\ell+1}\mathbf{f}_{i}^{\ell+1} + \mathbf{b}^{\ell+1}\big).
\]

So a full Psiformer layer \(\ell\) is:

\begin{enumerate}
\def\labelenumi{\arabic{enumi}.}
\tightlist
\item
  Self-attention: \(\{\mathbf{h}_i^\ell\} \to \{\mathbf{A}^\ell_i\}\).
\item
  Linear + residual:
  \(\{\mathbf{A}^\ell_i\} \to \{\mathbf{f}_i^{\ell+1}\}\).
\item
  MLP + residual:
  \(\{\mathbf{f}_i^{\ell+1}\} \to \{\mathbf{h}_i^{\ell+1}\}\).
\end{enumerate}

Repeat this for \(\ell=0,\dots,L-1\) and you get \textbf{final hidden
states} \[
\mathbf{h}_j^{L} \quad \text{for each electron } j.
\] Here: - \(L\) = number of layers in the Psiformer. - For each layer,
\(i\) indexes the electron the output belongs to, \(j\) indexes
electrons we attend over. - \(h\) indexes different heads in multi-head
attention.

\subsubsection{From hidden states to orbitals and
determinants}\label{from-hidden-states-to-orbitals-and-determinants}

From the final hidden states \(\mathbf{h}_j^L\), we build the
spin-orbital matrix for each determinant \(k\).

For each determinant index \(k\) and orbital index \(i\), define a
\textbf{linear ``orbital head''}: \[
\tilde{\phi}^{k}_i(x_j)
=
\mathbf{w}^{k}_i \cdot \mathbf{h}^{L}_j
+
g^{k}_i,
\] where \(\mathbf{w}^{k}_i\) and \(g^{k}_i\) are learned. The
dependence on spin \(\sigma_j\) and all other electrons is implicit in
\(\mathbf{h}_j^L\): the self-attention layers have already mixed that
information in.

Then we multiply by an \textbf{envelope} to enforce the correct
asymptotic decay: \[
\Omega^{k}_{ij}
=
\sum_{m}
\pi^{k}_{im}
\exp\big(
- \big|\mathbf{\Sigma}^{k}_{im}(\mathbf{r}_j - \mathbf{R}_m)\big|
\big),
\] where - \(m\) indexes nuclei (or ``envelope centers''), -
\(\pi^{k}_{im}\) and \(\mathbf{\Sigma}^{k}_{im}\) are learned
parameters.

The final spin-orbital entries are \[
\Phi^{k}_{ij}
=
\Omega^{k}_{ij}\,
\tilde{\phi}^{k}_i(x_j).
\]

Collecting these into the matrix \[
\boldsymbol{\Phi}^k(\mathbf{x}) = 
\big[\Phi^{k}_{ij}\big]_{i,j=1}^N,
\] we form the determinant \[
\det[\boldsymbol{\Phi}^k(\mathbf{x})]
=
\det\big[\Phi^{k}_{ij}\big]
=
\det\big[\phi^{k}_i(x_j)\big],
\] and finally the full Psiformer wavefunction \[
\Psi_{\theta}(\mathbf{x})
=
\exp(\mathcal{J}_{\theta}(\mathbf{x}))
\sum_{k=1}^{N_{\det}}\det[\boldsymbol{\Phi}^{k}_{\theta}(\mathbf{x})].
\]

So the story in terms of indices is:

\begin{itemize}
\tightlist
\item
  \(i\) = which \textbf{orbital} (row of the determinant).
\item
  \(j\) = which \textbf{electron} the orbital is evaluated on (column of
  the determinant).
\item
  \(k\) = which \textbf{determinant} in the sum.
\item
  \(\ell\) = which \textbf{layer} of the attention/MLP stack produced
  the hidden states.
\item
  \(h\) = which \textbf{attention head} participated in mixing
  information across electrons.
\item
  \(I,m\) = which \textbf{nucleus/center} is used for the envelope.
\end{itemize}

The self-attention layers are what let \(\mathbf{h}_j^L\) depend on all
other electrons in a flexible, learned way, while the determinant over
\(i,j\) and the Jastrow over \(i,j\) enforce fermionic antisymmetry and
cusp conditions.

!{[}{[}psiformer.png\textbar271x339{]}{]}

(Architecture of Psi Former, Source: Pfau et all)

\section{Methodology}\label{methodology}

To implement the code, the choose of the library is important. The three
options to implement this kind of matter are JAX, Tensor Flow and
pytorch, each one with his advantages and disadvantages. \#\#
Environment

For this project we are going to be using Pytorch due his user-friendly
and support. Python, and several libraries as transformers from hugging
face, a library who implements KFCA and like guide the implement of
Fermi Net by google deepmind which is made it on TensorFlow

Project manager with UV. \#\# Training

Due the high computational power needed we are going to using GPUS and
of course CUDA. Is clear that we are going to use virtual GPUS, for that
matter we have two option or well use a GPU via SSH or directly using
services like Azure , Colab, or anothers matters. For simplicity we are
going to use Colab services. The election of the GPU is not trivial. use
TPUS are not a bad idea. \#\# References

Bahdanau, D., Cho, K., \& Bengio, Y. (2014). Neural machine translation
by jointly learning to align and translate. \emph{arXiv Preprint
arXiv:1409.0473}.

Dosovitskiy, A., Beyer, L., Kolesnikov, A., Weissenborn, D., Zhai, X.,
Unterthiner, T., Dehghani, M., Minderer, M., Heigold, G., Gelly, S.,
Uszkoreit, J., \& Houlsby, N. (2021). \emph{An image is worth 16x16
words: Transformers for image recognition at scale}.
\url{https://arxiv.org/abs/2010.11929}

Jumper, J., Evans, R., Pritzel, A., Green, T., Figurnov, M.,
Ronneberger, O., Tunyasuvunakool, K., Bates, R., Žı́dek, A., Potapenko,
A., \& others. (2021). Highly accurate protein structure prediction with
AlphaFold. \emph{Nature}, \emph{596}(7873), 583--589.

Luo, D., \& Clark, B. K. (2019). Backflow transformations via neural
networks for quantum many-body wave functions. \emph{Physical Review
Letters}, \emph{122}(22).
\url{https://doi.org/10.1103/physrevlett.122.226401}

Pfau, D., Spencer, J. S., Matthews, A. G. D. G., \& Foulkes, W. M. C.
(2020). Ab initio solution of the many-electron Schrödinger equation
with deep neural networks. \emph{Physical Review Research}, \emph{2}(3).
\url{https://doi.org/10.1103/physrevresearch.2.033429}

Qiao, Z., Welborn, M., Anandkumar, A., Manby, F. R., \& Miller, T. F.
(2020). OrbNet: Deep learning for quantum chemistry using
symmetry-adapted atomic-orbital features. \emph{The Journal of Chemical
Physics}, \emph{153}(12). \url{https://doi.org/10.1063/5.0021955}

Raissi, M., Perdikaris, P., \& Karniadakis, G. E. (2019).
Physics-informed neural networks: A deep learning framework for solving
forward and inverse problems involving nonlinear partial differential
equations. \emph{Journal of Computational Physics}, \emph{378},
686--707. \url{https://doi.org/10.1016/j.jcp.2018.10.045}

Shang, H., Guo, C., Wu, Y., Li, Z., \& Yang, J. (2025). Solving the
many-electron Schrödinger equation with a transformer-based framework.
\emph{Nature Communications}, \emph{16}(1), 8464.
\url{https://doi.org/10.1038/s41467-025-63219-2}

Vaswani, A., Shazeer, N., Parmar, N., Uszkoreit, J., Jones, L., Gomez,
A. N., Kaiser, L., \& Polosukhin, I. (2017). Attention is all you need.
\emph{Advances in Neural Information Processing Systems (NeurIPS)},
\emph{30}.

von Glehn, I., Spencer, J. S., \& Pfau, D. (2023). \emph{A
self-attention ansatz for ab-initio quantum chemistry}.
\url{https://arxiv.org/abs/2211.13672}

\begin{center}\rule{0.5\linewidth}{0.5pt}\end{center}

\section{Excerpt}\label{excerpt}

Transformers are monsters finding relations between what you give them.
Is tempting use them for emulate the relations between electrons and
protons. How you can first encode the information of the electron's
positions and second the attraction and repulsion between the particles?







\end{document}
