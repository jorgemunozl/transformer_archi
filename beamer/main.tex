\documentclass{beamer}
\usetheme{Madrid}
\usecolortheme{dolphin}
\usefonttheme{professionalfonts}
\usepackage{amsmath,amssymb,bm,mathtools,physics,siunitx,tikz,xcolor,graphicx,hyperref}
\title{Solving the Many-Electron Schr\"odinger Equation}
\subtitle{With a Transformer Architecture}
\author{Jorge Munoz}
\date{\today}
\begin{document}

\begin{frame}
    \titlepage
\end{frame}

\begin{frame}
    \frametitle{Outline}
    \tableofcontents
\end{frame}

%-----------------------

\section{The Schr\"odinger Wave Function and the physical laws that rule}

\subsection{Schr\"odinger Equation}

\begin{frame}{Many-electron Schr\"odinger equation}
    \begin{itemize}
        \item Wavefunction $\Psi(\mathbf r_1,\sigma_1,\dots,\mathbf r_N,\sigma_N;t)$ encodes all information about the system; $|\Psi|^2$ gives a probability density that integrates to 1.
        \item Time-dependent Schr\"odinger equation (TDSE):
        \[
            i\,\hbar\,\partial_t \Psi = H\,\Psi,
        \]
        where $H$ is the molecular Hamiltonian.
        \item In atomic units ($\hbar = m_e = e = 1$) with fixed nuclei at $\mathbf R_I$:
        \[
            H = -\tfrac{1}{2}\sum_{i=1}^N \nabla_i^2 - \sum_{i,I} \frac{Z_I}{r_{iI}} + \sum_{i<j} \frac{1}{r_{ij}}.
        \]
        \item Separating time $\Psi(\mathbf r, t) = e^{-iEt}\,\psi(\mathbf r)$ yields the time-independent equation (TISE):
        \[
            H\,\psi = E\,\psi,
        \]
        an eigenvalue problem whose lowest solution gives the ground-state energy.
    \end{itemize}
\end{frame}

\begin{frame}{Setup: distances, spins, coordinates for many body systems}
    
    \begin{itemize}
        \item Kinetic energy: $T = -\frac{1}{2}\sum_{i=1}^{N} \nabla_i^2$.
        \item Electron-nuclear attraction: $V_{en} = -\sum_{i,I} \frac{Z_I}{r_{iI}}$.
        \item Electron-electron repulsion: $V_{ee} = \sum_{i<j} \frac{1}{r_{ij}}$.
        \item Hamiltonian: $H = T + V_{en} + V_{ee}$.
        \item Solve the time-independent Schr\"odinger equation: $H\,\Psi = E\,\Psi$.
    \end{itemize}
    
    \begin{itemize}
        \item System: $N$ electrons, nuclei with charge $Z_I$ at fixed $\mathbf R_I$.
        \item Electron coordinates $\mathbf r_i \in \mathbb{R}^3$, spin $\sigma_i \in \{\uparrow,\downarrow\}$.
        \item Atomic units (distances in Bohr, energies in Hartree).
        \item Distances: $r_{ij} = |\mathbf r_i - \mathbf r_j|$, $r_{iI} = |\mathbf r_i - \mathbf R_I|$.
    \end{itemize}
\end{frame}

\subsection{Conditions}

Physical Laws

\begin{frame}{Fermi--Dirac statistics \& antisymmetry}
    \begin{itemize}
        \item Electrons are indistinguishable fermions.
        \item Exchanging two electrons flips the wavefunction's sign: $\Psi(\dots i,j \dots) = -\,\Psi(\dots j,i \dots)$.
        \item Pauli exclusion: no two electrons can occupy the same state.
        \item Use a Slater determinant to enforce an antisymmetric $\Psi$.
    \end{itemize}
\end{frame}

\begin{frame}{Exponential decay of wavefunction}
    \begin{itemize}
        \item Bound-state wavefunctions decay exponentially as $r \to \infty$.
        \item Example: hydrogen ground state $\psi(r) \sim e^{-r}$ (in a.u.).
        \item Ans\"atze include an exponential envelope so that $\Psi \to 0$ as $r \to \infty$.
    \end{itemize}

\end{frame}

\begin{frame}{Kato cusp conditions, Jastrow Factor}
    \begin{itemize}
        \item Coulomb potentials cause a sharp cusp in $\Psi$ when particles coalesce.
        \item Electron--nucleus cusp: $\displaystyle \frac{\partial \Psi}{\partial r_{iI}}\Big|_{r_{iI}=0} = -\,Z_I\,\Psi(0)$.
        \item Electron--electron cusp: $\displaystyle \frac{\partial \Psi}{\partial r_{ij}}\Big|_{r_{ij}=0} = \frac{1}{2}\,\Psi(0)$.
    \end{itemize}
\end{frame}


% ------------------
\section{Deep Learning Tool Box}

% Natural Gradient Descent

\begin{frame}{The Psiformer Ansatz (overview)}
    \begin{itemize}
        \item Neural network trial wavefunction (``Wavefunction Transformer'').
        \item Uses self-attention layers to model electron correlations.
        \item Permutation equivariant (independent of electron ordering).
        \item Greatly improves accuracy, especially for larger systems.
        \item Solves Schr\"odinger equation from first principles (no external data).
    \end{itemize}

\end{frame}

\begin{frame}{Dissecting the Ansatz}
    \begin{itemize}
        \item Slater--Jastrow form: $\Psi = \Big(\sum\limits_d c_d\,\det[\phi^d_k(\mathbf r_i)]\Big)\,\exp(J)$.
        \item Slater determinant part ensures the correct antisymmetric exchange.
        \item One-electron orbitals $\phi^d_k$ are learned functions of all electron coordinates.
        \item Jastrow factor $J$ enforces electron-electron cusp conditions.
        \item Additional envelope functions ensure exponential decay at long range.
    \end{itemize}
\end{frame}

\begin{frame}{Jastrow factor form (parallel/antiparallel spins)}
    \begin{itemize}
        \item Jastrow factor: $J = \sum_{i<j} u_{\sigma_i,\sigma_j}(r_{ij})$.
        \item Use separate $u(r)$ for same-spin vs opposite-spin pairs.
        \item Example choice: $u_{\uparrow\downarrow}(r) = \frac{\alpha\,r}{1 + \beta\,r}$ (two parameters).
        \item Satisfies cusp: $u(r) \approx \frac{1}{2}r$ as $r \to 0$.
        \item Jastrow adds correlation beyond antisymmetry (lowers the energy).
    \end{itemize}
\end{frame}

\begin{frame}
    \frametitle{Loss: Rayleigh quotient}
    \begin{itemize}
        \item Variational principle uses the energy expectation as loss.
        \item $E[\Psi] = \displaystyle \frac{\langle \Psi \mid H \mid \Psi \rangle}{\langle \Psi \mid \Psi \rangle} \ge E_0$ (Rayleigh quotient).
        \item Minimizing $E[\Psi]$ drives the ansatz toward the ground state.
        \item $E_0$ is the true ground energy (minimum possible).
    \end{itemize}
\end{frame}

\begin{frame}{Metropolis--Hastings sampling}
    \begin{itemize}
        \item Markov Chain Monte Carlo algorithm to sample $|\Psi(\mathbf r)|^2$.
        \item Propose random moves for electron coordinates.
        \item Accept move with probability $A = \min\!\Big(1,\; \frac{|\Psi_{\text{new}}|^2}{|\Psi_{\text{old}}|^2}\Big)$.
        \item After thermalization, obtained samples follow the target distribution $|\Psi|^2$.
    \end{itemize}
\end{frame}

\begin{frame}
    \frametitle{Variational Monte Carlo loop}
    \begin{itemize}
        \item Initialize neural network parameters (trial $\Psi$).
        \item Sample electron configurations via Metropolis--Hastings.
        \item Evaluate energy and gradient from the sampled configurations.
        \item Update parameters to lower the energy (gradient descent).
        \item Iterate until convergence to the minimum energy (ground state).
    \end{itemize}
\end{frame}

\begin{frame}
    \frametitle{Optimizer: KFAC (natural gradient)}
    \begin{itemize}
        \item Kronecker-Factored Approximate Curvature (KFAC) optimizer.
        \item Approximates the natural gradient using a factored curvature matrix.
        \item Speeds up training and improves stability for large networks.
        \item Used to efficiently optimize FermiNet and Psiformer wavefunctions.
    \end{itemize}
\end{frame}

\begin{frame}
    \frametitle{FermiNet baseline}
    \begin{itemize}
        \item FermiNet: first deep-neural-network wavefunction ansatz (Pfau et al., 2020).
        \item Architecture: multiple dense layers with electron-wise feature streams.
        \item Outputs single-electron orbitals feeding into a Slater determinant.
        \item Achieved high accuracy on small molecules (near chemical accuracy).
        \item Faced scaling limits: accuracy and efficiency drop for larger systems.
    \end{itemize}
\end{frame}

\begin{frame}
    \frametitle{Psiformer vs FermiNet: Architecture \& Attention}
    \begin{itemize}
        \item FermiNet layers mix electron features via fixed functions; Psiformer uses self-attention.
        \item Self-attention: each electron attends to all others (learns interactions).
        \item Both ansatzes enforce antisymmetry via Slater determinants.
        \item Psiformer captures correlations more effectively with fewer parameters.
        \item Attention mechanism is permutation-invariant and scales to complex interactions.
    \end{itemize}
\end{frame}

\section{Fermi Net and Psiformer}

\begin{frame}
    \frametitle{Computational power \& scaling}
    \begin{itemize}
        \item Solving many-electron systems is computationally intensive.
        \item Computational cost grows steeply with electron number $N$.
        \item Wavefunction evaluation involves expensive operations (e.g. determinants, $O(N^3)$).
        \item Requires significant computing resources (GPUs/TPUs, parallelization).
    \end{itemize}
\end{frame}

\begin{frame}
    \frametitle{Numerical stability}
    \begin{itemize}
        \item Exponential and cusp factors can cause numerical overflow/underflow if not handled carefully.
        \item Ensuring stable Monte Carlo estimates (variance reduction techniques).
        \item Feature scaling and proper initialization improve training stability.
        \item Jastrow factor helps manage large energy fluctuations (by satisfying cusps).
    \end{itemize}
\end{frame}

\begin{frame}
    \frametitle{Scaling with electron count}
    \begin{itemize}
        \item Wavefunction resides in a $3N$-dimensional configuration space.
        \item Monte Carlo integration becomes harder as $N$ increases (high dimensionality).
        \item More electrons often require deeper or wider neural networks.
        \item Ongoing research into architectures that scale better or transfer to larger $N$.
    \end{itemize}
\end{frame}

\begin{frame}
    \frametitle{Thanks}
    \centering
    Thank you for your attention!
\end{frame}

\end{document}
